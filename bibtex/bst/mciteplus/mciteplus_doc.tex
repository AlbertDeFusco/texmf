%% mciteplus_doc.tex
%% V1.1
%% 2008/09/30
%% Copyright (c) 2008 by Michael Shell
%% See:
%% http://www.michaelshell.org/
%% for current contact information.
%% 
%% This is the documentation for the mciteplus.sty package.
%% 
%% Support sites:
%% http://www.michaelshell.org/tex/mciteplus/
%% http://www.ctan.org/tex-archive/macros/latex/contrib/mciteplus/
%%
%%*************************************************************************
%% Legal Notice:
%% This code is offered as-is without any warranty either expressed or
%% implied; without even the implied warranty of MERCHANTABILITY or
%% FITNESS FOR A PARTICULAR PURPOSE! 
%% User assumes all risk.
%% In no event shall IEEE or any contributor to this code be liable for
%% any damages or losses, including, but not limited to, incidental,
%% consequential, or any other damages, resulting from the use or misuse
%% of any information contained here.
%%
%% All comments are the opinions of their respective authors and are not
%% necessarily endorsed by the IEEE.
%%
%% This work is distributed under the LaTeX Project Public License (LPPL)
%% ( http://www.latex-project.org/ ) version 1.3, and may be freely used,
%% distributed and modified. A copy of the LPPL, version 1.3, is included
%% in the base LaTeX documentation of all distributions of LaTeX released
%% 2003/12/01 or later.
%% Retain all contribution notices and credits.
%% ** Modified files should be clearly indicated as such, including  **
%% ** renaming them and changing author support contact information. **
%%
%% File list of work: mciteplus.sty, mciteplus_doc.pdf, mciteplus_doc.tex,
%%                    mciteplus_code.txt, apsrevM.bst, apsrmpM.bst, 
%%                    IEEEtranM.bst, IEEEtranMN.bst
%%*************************************************************************

\documentclass[10pt,letterpaper]{article}


% tweak article.cls
\makeatletter

\newcommand\tableofcontentssectiononly{%
    \section*{\contentsname
        \@mkboth{%
           \MakeUppercase\contentsname}{\MakeUppercase\contentsname}}}
\newcommand\tableofcontentsstarttoc{\@starttoc{toc}}


\def\@maketitle{%
  \newpage
  \null
  \vspace{-3em}%
  \begin{center}%
  \let \footnote \thanks
    {\LARGE \@title \par}%
    \vskip 1em%
    {\large
      \lineskip .5em%
      \begin{tabular}[t]{c}%
        \@author
      \end{tabular}\par}%
    \vskip 1em%
    {\large \@date}%
  \end{center}
  \par
  \vskip 1em}

\makeatother



\usepackage{multicol}



\parskip 1ex plus 1ex minus 0.75ex

\parindent    1.0em

\topmargin    -49.0pt
\headheight   12pt
\headsep      0.25in
\footskip     0.4in

% want 1in margins
\setlength{\topmargin}{-\headsep}%
\addtolength{\topmargin}{-\headheight}%
% we want 1in side margins regardless of paper type
\oddsidemargin      0in
\evensidemargin     0in
% set the text width
\setlength{\textwidth}{\paperwidth}%
\addtolength{\textwidth}{-2.0in}%
\setlength{\textheight}{\paperheight}%
\addtolength{\textheight}{-2.0in}%
% digitize textheight to be an integer number of lines.
% this may cause the bottom margin to be off a tad
\addtolength{\textheight}{-1\topskip}%
\divide\textheight  by \baselineskip%
\multiply\textheight  by \baselineskip%
\addtolength{\textheight}{\topskip}%


\newcommand{\MYdocversion}{1.1}

\usepackage{type1ec}
%\usepackage{type1cm}
\usepackage[T1]{fontenc}

\usepackage{ifpdf}
\usepackage{notoccite}

%\usepackage[retainorgcmds]{IEEEtrantools}
% use roman url style
\usepackage{url}

\csname url@rmstyle\endcsname
\ifpdf
  \usepackage[pdftex]{graphicx}
  \graphicspath{{.}}
  \DeclareGraphicsExtensions{.pdf}
\else
  \usepackage[dvips]{graphicx}
  \graphicspath{{.}}
  \DeclareGraphicsExtensions{.eps}
\fi

\ifpdf
  \usepackage[pdftex]{thumbpdf}
\else
  \usepackage[dvips]{thumbpdf}
\fi

\newcommand\MYhyperrefoptions{hypertexnames=false,bookmarks=true,
bookmarksnumbered=true, pdfpagemode={UseOutlines},plainpages=false,
pdfpagelabels=true,colorlinks=true,linkcolor={darkred},
citecolor={darkgreen},pagecolor={black},urlcolor={darkblue},
pdftitle={Mciteplus: Enhanced Multicitations},
pdfsubject={Typesetting},
pdfauthor={Michael D. Shell},
pdfkeywords={bibliography, BibTeX, collapsed entries, LaTeX, mcite,
mciteplus, multicitations, physics journals, references, typesetting}}
\ifpdf
  \usepackage[\MYhyperrefoptions,pdftex]{hyperref}
\else
  \usepackage[\MYhyperrefoptions,dvips]{hyperref}
  \usepackage{breakurl}
\fi

\usepackage{color}
\definecolor{darkred}{rgb}{0.8,0.0,0.0}
\definecolor{darkgreen}{rgb}{0.0,0.5,0.0}
\definecolor{darkblue}{rgb}{0.0,0.0,0.7}


\providecommand{\texorpdfstring}[2]{#1}


% Improved \textunderscore to provide a much better fake _ when used with
% OT1 encoding. Under OT1, detect use of pcr or cmtt \ttfamily and use
% available true _ glyph for those two typewriter fonts.
\makeatletter
\def\@MYstringptm{ptm} % Times Roman family
\def\@MYstringppl{ppl} % Palatino Roman family
\def\@MYstringphv{phv} % Helvetica Sans Serif family
\def\@MYstringpcr{pcr} % Courier typewriter family
\def\@MYstringcmtt{cmtt} % Computer Modern typewriter family
\DeclareTextCommandDefault{\textunderscore}{\leavevmode
\ifx\f@family\@MYstringpcr\string_\else
\ifx\f@family\@MYstringcmtt\string_\else
\ifx\f@family\@MYstringptm\kern 0em\vbox{\hrule\@width 0.5em\@height 0.5pt\kern -0.3ex}\else
\ifx\f@family\@MYstringppl\kern 0em\vbox{\hrule\@width 0.5em\@height 0.5pt\kern -0.3ex}\else
\ifx\f@family\@MYstringphv\kern -0.03em\vbox{\hrule\@width 0.62em\@height 0.52pt\kern -0.33ex}\kern -0.03em\else
\kern 0.09em\vbox{\hrule\@width 0.6em\@height 0.44pt\kern -0.63pt\kern -0.42ex}\kern 0.09em\fi\fi\fi\fi\fi\relax}
\makeatother

\makeatletter
\def\bstctlcite{\@ifnextchar[{\@bstctlcite}{\@bstctlcite[@auxout]}}
\def\@bstctlcite[#1]#2{\@bsphack
  \@for\@citeb:=#2\do{%
    \edef\@citeb{\expandafter\@firstofone\@citeb}%
    \if@filesw\immediate\write\csname #1\endcsname{\string\citation{\@citeb}}\fi}%
  \@esphack}
\makeatother

\makeatletter
%% ***** Start ZAX defs *****
%% based on unreleased zax.sty, not ready yet for formal use
%% V0.9
%% 2005/05/02
%% Copyright (c) 2005 by Michael Shell
%% see http://www.michaelshell.org/contact.html
%% for current contact information.
%
%
% Showing LaTeX source code poses some difficulties. I probably could have
% done things with \verbatim. But, I don't like it for a number of reasons.
% For me, the main drawbacks are: 
%
% 1. hyphenates words 
% 2. does not perfectly left and right justify the \ttfamily text
% 3. does not allow the use of active LaTeX commands within the displayed
%    code
%
% Basically, I wanted a little more control.
%
% So, I developed a set of commands that will display LaTeX code. 
% The \ZAXlong and \ZAXshort commands allow the direct
% display of LaTeX commands. The difference between the two is that
% \ZAXlong will retain spaces and allows the inter-letter spacing
% to expand slightly so that the text will always be justified while
% \ZAXshort ignores spaces and uses no inter-letter space.
% Variations of these two can be created by the user to provide
% fo different font sizes and styles.
%
% When one of these commands is issued:
%
% |    becomes the control character: |relax 
%      ( use |ZAXbar to get a | character)
%
% { }  becomes like punctuation 
%      (use |bgroup and |egroup instead) 
%
% +    prevents a break between the characters it appears between
%      (use |ZAXplus to get a + character)
%
% ~    nonbreaking spaces become visible tildes
% 
% breaks are not allowed right after a "\" or a space
%
% For the long form, since normal spaces are not ignored:
% &    becomes an (ignored) space character for use after commands.
%      e.g., " |mbox|bgroup&Unbro|egroup&ken and "
%
% to stop the action of a \ZAX command, issue an |end.
%
% The trick is to read all the input characters and to insert a "sliver"
% of hspace glue between them! TeX will be happy to make a break at any
% glue, and it will never hyphenate at the break. With the long form, 
% this interletter glue is also allowed to stretch a very small amount -
% because a long sequence could stretch across the entire column without
% any spaces to break on.
% When real spaces are encountered, we replace them with \hspace* commands
% of the same nominal length, but that can stretch and shrink a fair
% amount. The \hspace* form ensures that the space will not be lost
% at the start/end of the lines. By using \penalty10000 at the end of
% the artificial spaces, we ensure that TeX will not break right after a
% space - the user will be able to explicitly see the space at the
% beginning of the next line.
%
% This command family can be thought of as implementing a type of "backward"
% line breaking world in which breaks are assumed to possible everywhere except
% were indicated as prohibited. Furthermore, hyphenation is not used as spaces
% are not implicit with line breaks, but must be carried to the start of the
% next line. In other words, commands are not thought of as words, but rather
% as sequences of letters.
%
%
% Note: TeX parser tokenizes commands and special characters when acquiring
% an argument for a command. Thus, when used within an argument to a command,
% the code in the \ZAX never gets a chance to alter "\", "|", "{", "}",
% etc., before TeX acquires the argument. For such use, the syntax is reversed,
% i.e., call the commands using the "\" as one normally would. The "+", "~" will
% still operate as described above:
% \footnote{\ZAXlong this\ZAXplus is+a\ZAXbar big\_\ZAXbsl mbox%
% \ZAXobrace~\ZAXcbrace\ test\end.}
% Future versions of this code may use this latter approach exclusively so
% as to unify the syntax (although entering LaTeX code will then take longer
% command sequences in the main text).


\def\ZAXplus{\mbox{+}}  % a plus sign
\def\ZAXbar{\mbox{|}}   % a vert bar
\def\ZAXvtilde{\~{}}    % a visible tilde
%\def\ZAXvtilde{\textasciitilde} % another form of a visible tilde

% define \ZAXbsl which is a "\" character 
\catcode`\|=0 \catcode`\\=12
|edef|ZAXbsl{|mbox{\}}
|catcode`|\=0 |catcode`||=12

% define \ZAXbslnobrk macro that gives an unbreakable backslash character
% this is used only in \ZAXrun to prevent breaks around \ 
\catcode`\|=0\catcode`\\=12 |def|ZAXbslnobrk{\} |catcode`|\=0 |catcode`||=12

% define \ZAXobrace which is a "{" character 
\catcode`\$=1 \catcode`\{=12
\edef\ZAXobrace$\mbox${}}
\catcode`\$=3 \catcode`\{=1

% define \ZAXcbrace which is a "}" character 
\catcode`\$=2 \catcode`\}=12
\edef\ZAXcbrace{\mbox{}$$
\catcode`\$=3 \catcode`\}=2

% define \ZAXmath which is the "$" math shift character
\edef\ZAXmath{$}

% makes spaces visible
\def\ZAXvspon{\catcode`\ =12}
% makes spaces invisible
\def\ZAXvspoff{\catcode`\ =10}

% \ZAXvsp is a macro that produces a visible space
% can be used to show spaces within the short or long cmdword forms
\ZAXvspon\def\ZAXvsp{ }\ZAXvspoff

% \\ does not work in ZAX environments do to the change in
% cat codes. We provide it here as \ZAXnl - this definition
% is updated as ZAX starts.
\let\ZAXnl=\\

% fat space nominal factor (multiplied by width of \ )
\def\ZAXfatspacenorm{1}
% fat space stretch factor
\def\ZAXfatspacestretch{0.43}
% fat space shrink factor
\def\ZAXfatspaceshrink{0.24}
% slim space
\def\ZAXslimspacelen{0pt plus 0.15pt minus 0pt}



% these are determined in the start code, but provide dummys here so they are defined
\def\ZAXfatspace{\hspace*{10pt}\penalty10000}
\def\ZAXslimspace{\penalty0\hspace{0pt plus 0pt minus 0pt}}


% flag to indicate if font was saved
\newif\if@ZAXsavedfont       \@ZAXsavedfontfalse

% saves all the properties of the given font
% use global in case font save is set within a group
\def\ZAXsavefont{\global\let\@ZAXfontENCODINGsave\f@encoding%
\global\let\@ZAXfontFAMILYsave\f@family\global\let\@ZAXfontSERIESsave\f@series%
\global\let\@ZAXfontSHAPEsave\f@shape\global\let\@ZAXfontSIZEsave\f@size%
\global\let\@ZAXfontBASELINESKIPsave\f@baselineskip\global\@ZAXsavedfonttrue}
% restores font back to saved value
\def\ZAXrestorefont{\fontencoding{\@ZAXfontENCODINGsave}\fontfamily{\@ZAXfontFAMILYsave}%
\fontseries{\@ZAXfontSERIESsave}\fontshape{\@ZAXfontSHAPEsave}%
\fontsize{\@ZAXfontSIZEsave}{\@ZAXfontBASELINESKIPsave}\selectfont}


% lengths
\newskip\@ZAXtemplen
\newskip\@ZAXfatspacelen


% Setup for the longcmdword form
% make a stretchable fat spacer equal in length to the nominal space length
% use penalty's to ensure easy breaks before and impossible breaks after spaces
% allow the inter-letter slim "sliver" space to stretch a tad
% make | the control character, & a space, and \{}, $ and blank characters
% equal to punctuation
\def\ZAXstartLong{\let\@ZAXnextbrk=0% no sliver before start
\settowidth{\@ZAXfatspacelen}{\ }%
\setlength{\@ZAXfatspacelen}{\ZAXfatspacenorm\@ZAXfatspacelen}%
\setlength{\@ZAXtemplen}{\ZAXfatspacestretch\@ZAXfatspacelen}%
\addtolength{\@ZAXfatspacelen}{0pt plus \the\@ZAXtemplen}%
\setlength{\@ZAXtemplen}{\ZAXfatspaceshrink\@ZAXfatspacelen}%
\addtolength{\@ZAXfatspacelen}{0pt minus \the\@ZAXtemplen}%
\def\ZAXfatspace{\penalty0\hspace*{\@ZAXfatspacelen}\penalty10000}%
\def\ZAXslimspace{\hspace{\ZAXslimspacelen}}%
\let\ZAXnl=\\%
\catcode`\|=0 \catcode`\\=12 \catcode`\{=12 \catcode`\}=12 \catcode`\ =12 \catcode`\&=10%
\catcode`\^=12 \catcode`\$=12 \catcode`\#=12%
\ZAXrun}


% Setup for the shortcmdword form
% make a stretchable fat spacer equal in length to the nominal space length
% use penalty's to ensure impossible breaks after spaces
% the inter-letter slim "sliver" space is rigid at 0pt
% make | the control character and \{}, $, # and characters equal to punctuation
\def\ZAXstartShort{\let\@ZAXnextbrk=0%% no sliver before start
\settowidth{\@ZAXfatspacelen}{\ }%
\setlength{\@ZAXtemplen}{\ZAXfatspacestretch\@ZAXfatspacelen}%
\addtolength{\@ZAXfatspacelen}{0pt plus \the\@ZAXtemplen}%
\setlength{\@ZAXtemplen}{\ZAXfatspaceshrink\@ZAXfatspacelen}%
\addtolength{\@ZAXfatspacelen}{0pt minus \the\@ZAXtemplen}%
\def\ZAXfatspace{\hspace*{\@ZAXfatspacelen}\penalty10000}%
\def\ZAXslimspace{\hspace{0pt plus 0pt minus 0pt}}%
\let\ZAXnl=\\%
\catcode`\|=0 \catcode`\\=12 \catcode`\{=12 \catcode`\}=12 \catcode`\$=12 \catcode`\#=12%
\ZAXrun}


\def\ZAXrun#1{\ifx#1\end\catcode`\{=1 \catcode`\}=2 \catcode`\\=0% stop if we hit an |end
\catcode`\|=12 \catcode`\ =10 \catcode`\&=4 \catcode`\^=7 \catcode`\$=3 \catcode`\#=6%
\if@ZAXsavedfont\ZAXrestorefont\fi\global\@ZAXsavedfontfalse\let\@ZAXnext=\relax%
\else%
\def\@ZAXtokenout{#1}% save the token as a macro
\let\@ZAXout=1\relax% assume we wish to output this character
\let\@ZAXbrk=1\relax% assume this is a breakable point
\ifx\@ZAXnextbrk0\let\@ZAXbrk=0\fi% unless prev char wants no break here
\let\@ZAXnextbrk=1\relax% we assume this char will allow the next char to break
\let\@ZAXoutspace=\ZAXslimspace% assume the output spacer is a sliver
\ifx\@ZAXtokenout\ZAXvsp\let\@ZAXout=0\let\@ZAXoutspace=\ZAXfatspace\let\@ZAXnextbrk=0\fi% visible space, output the space instead
\ifx#1\ \let\@ZAXout=0\let\@ZAXoutspace=\ZAXfatspace\let\@ZAXnextbrk=0\fi% control space, output the spacer instead
\ifx#1+\let\@ZAXout=0\let\@ZAXbrk=0\let\@ZAXnextbrk=0\fi% a + means not to break here or at the next, discard token and no spacer
\ifx\@ZAXtokenout\ZAXbslnobrk\let\@ZAXnextbrk=0\fi% we don't want breaks right after \
\ifx#1~\def\@ZAXtokenout{\ZAXvtilde}\fi% make nonbreaking spaces visible
%\ifx#1\ZAXobrace\def\@ZAXtokenout{{\catcode`\{=1\{}}\typeout{obrace}\fi%
%\ifx#1\ZAXcbrace\def\@ZAXtokenout{\}}\fi%
\ifx\@ZAXbrk1\@ZAXoutspace\else\penalty10000\relax\@ZAXoutspace\fi% output spacer, stop breaks if needed
\ifx\@ZAXout1\ifx\@ZAXbrk1\relax\@ZAXtokenout\relax\else\penalty10000\relax\@ZAXtokenout\relax\fi\fi% output the token if desired 
\let\@ZAXnext=\ZAXrun% ready to get next token
\fi% now get the next token
\@ZAXnext}


\def\ZAXlong#1{\ZAXsavefont#1\ZAXstartLong}
\def\ZAXshort#1{\ZAXsavefont#1\ZAXstartShort}


% to define custom commands, do like:
% \def\ZAXlongSTT{\ZAXsavefont\small\ttfamily\ZAXstartLong}


%% ***** End ZAX defs *****
\makeatother


% Bogus thebibliography for apsrev.bst extracts
\makeatletter
\newenvironment{MYapsrevthebibliography}[1]{\relax
\list{\@biblabel{\@arabic\c@enumiv}}%
           {\settowidth\labelwidth{\@biblabel{#1}}%
            \leftmargin\labelwidth
            \advance\leftmargin\labelsep
            \@openbib@code
            \usecounter{enumiv}%
            \let\p@enumiv\@empty
            \renewcommand\theenumiv{\@arabic\c@enumiv}}%
      \sloppy
      \clubpenalty4000
      \@clubpenalty \clubpenalty
      \widowpenalty4000%
      \sfcode`\.\@m\relax
\expandafter\ifx\csname bibnamefont\endcsname\relax
  \def\bibnamefont##1{##1}\fi
\expandafter\ifx\csname bibfnamefont\endcsname\relax
  \def\bibfnamefont##1{##1}\fi
\expandafter\ifx\csname citenamefont\endcsname\relax
  \def\citenamefont##1{##1}\fi
\expandafter\ifx\csname url\endcsname\relax
  \def\url##1{\texttt{##1}}\fi
\expandafter\ifx\csname urlprefix\endcsname\relax\def\urlprefix{URL }\fi
\providecommand{\bibinfo}[2]{##2}\relax
\providecommand{\eprint}[2][]{\url{##2}}\relax
\def\@MYapsrevbogusbibitem{\@ifnextchar[{\@@MYapsrevbogusbibitem}{\@@MYapsrevbogusbibitem[\relax]}}\relax
\def\@@MYapsrevbogusbibitem[##1]##2{\item}\relax
\let\bibitem\@MYapsrevbogusbibitem}{\endlist}
\makeatother


\def\BibTeX{{\rmfamily B\kern-.05em{\scshape i\kern-.025em b}\kern-.08em
    T\kern-.1667em\lower.7ex\hbox{E}\kern-.125emX}}

% needed only when not using the CM super fonts as they do not
% have bold small caps needed for a bold \BibTeX
%\newcommand{\MYfontfix}[1]{\textmd{#1}}
%
\newcommand{\MYfontfix}[1]{\relax #1}

\newenvironment{MYlistblk}{\parindent=0pt\trivlist\item\relax\small}{\endtrivlist}

\newcommand{\MYemdash}{\,---\,}

\newcommand{\MYshortcmd}{\ZAXshort{\ttfamily\small}}
\newcommand{\MYlongcmd}{\ZAXlong{\ttfamily\small}}

\newcommand{\MYsmalltt}[1]{\texttt{\small #1}}

\newcommand{\MYsecref}[1]{section~\ref{#1}}
\newcommand{\MYSecref}[1]{Section~\ref{#1}}


\pagestyle{headings}

%\vskip -20em


\title{Mciteplus: Enhanced Multicitations}
\author{Michael Shell\thanks{See \url{http://www.michaelshell.org/} for
current contact information.\hfil\break
\indent\ Manuscript originally created on January 15, 2008; revised September 30, 2008.
The latest version of this package can be obtained at \mbox{CTAN} \cite{ctan:mciteplus}.
This work is distributed under the \LaTeX\ Project Public License (LPPL)
 ( \url{http://www.latex-project.org/} )
version 1.3. A copy of the LPPL, version 1.3, is included in the base \LaTeX\ 
documentation of all distributions of \LaTeX\ released 2003/12/01 or later.
The opinions expressed here are entirely that of the author. No warranty is expressed
or implied. User assumes all risk.}\\
{\small\hfill\textit{with special thanks to Joseph Wright}\hfill\mbox{}}}

\date{Version 1.1, September 30, 2008\\[0.5\baselineskip]
{\normalfont\normalsize\url{http://www.michaelshell.org/tex/mciteplus/}}}


\flushbottom

\begin{document}

\bstctlcite{BSTcontrol}

\thispagestyle{empty}

\maketitle
\begin{abstract}
This \LaTeXe{} package is an enhanced reimplementation of Thorsten Ohl's mcite package
which provides support for the grouping of multiple citations together as is
often done in physics journals. An extensive set of features provide for
other applications such as reference sublisting.
\end{abstract}

\tableofcontentssectiononly
\begin{multicols}{2}
\tableofcontentsstarttoc
\end{multicols}



\section{Introduction}
Mciteplus is a reimplementation of Thorsten Ohl's
(\url{http://theorie.physik.uni-wuerzburg.de/~ohl/}) mcite \LaTeX\ package
\cite{ctan:mcite}, which
provides for ``collapsed'' citations (i.e., a grouping of multiple references
under a single collective reference) as is often done in physics journals.
Using Thorsten Ohl's example in the documentation for mcite:
\begin{MYapsrevthebibliography}{3}
\bibitem[{\citenamefont{Glashow}(1961)}]{Glashow}
\bibinfo{author}{\bibfnamefont{S.~L.} \bibnamefont{Glashow}},
  \bibinfo{journal}{Nucl. Phys.} \textbf{\bibinfo{volume}{22}},
  \bibinfo{pages}{579} (\bibinfo{year}{1961}).

\bibitem[{\citenamefont{Salam}(1968)}]{Salam}
\bibinfo{author}{\bibfnamefont{A.}~\bibnamefont{Salam}}, in
  \emph{\bibinfo{booktitle}{Elementary Particle Theory}}, edited by
  \bibinfo{editor}{\bibfnamefont{N.}~\bibnamefont{Svartholm}}
  (\bibinfo{publisher}{Almquist and Wiksell}, \bibinfo{address}{Stockholm},
  \bibinfo{year}{1968}), pp. \bibinfo{pages}{367--377}.

\bibitem[{\citenamefont{Weinberg}(1967)}]{Weinberg}
\bibinfo{author}{\bibfnamefont{S.}~\bibnamefont{Weinberg}},
  \bibinfo{journal}{Phys. Rev. Lett.} \textbf{\bibinfo{volume}{19}},
  \bibinfo{pages}{1264} (\bibinfo{year}{1967}).
\end{MYapsrevthebibliography}
becomes:\footnote{Formatting produced by the apsrevM.bst bibstyle.}\relax
\begin{MYapsrevthebibliography}{1}
\bibitem[{\citenamefont{Glashow}(1961)}]{Glashow}
\bibinfo{author}{\bibfnamefont{S.~L.} \bibnamefont{Glashow}},
  \bibinfo{journal}{Nucl. Phys.} \textbf{\bibinfo{volume}{22}},
  \bibinfo{pages}{579} (\bibinfo{year}{1961});
\bibinfo{author}{\bibfnamefont{A.}~\bibnamefont{Salam}}, in
  \emph{\bibinfo{booktitle}{Elementary Particle Theory}}, edited by
  \bibinfo{editor}{\bibfnamefont{N.}~\bibnamefont{Svartholm}}
  (\bibinfo{publisher}{Almquist and Wiksell}, \bibinfo{address}{Stockholm},
  \bibinfo{year}{1968}), pp. \bibinfo{pages}{367--377};
\bibinfo{author}{\bibfnamefont{S.}~\bibnamefont{Weinberg}},
  \bibinfo{journal}{Phys. Rev. Lett.} \textbf{\bibinfo{volume}{19}},
  \bibinfo{pages}{1264} (\bibinfo{year}{1967}).
\end{MYapsrevthebibliography}


\subsection{Features}
Mciteplus offers the following features many of which are not available with the
original mcite:
\begin{itemize}
\item Entry punctuation can be controlled by the bibstyle (.bst) as well as the user.
\item No ``double periods'' when an entry already ends with a period (e.g., an
abbreviated journal name).
\item Support for sublists.
\item Support for multiple bibliographies and/or auxiliary files.
\item The bibliography sample label width is automatically updated to account for the changes
      in the numbering due to the grouped entries. Maximum label width information is
      available to user.
\item Compatible with the natbib package. (The bibstyle must support both
mciteplus as well as natbib.)
\item Compatible with the REV\TeX4 class. (The bibstyle must support both
mciteplus as well as natbib.)
\item Support for the optional argument of \MYshortcmd\cite[]{}|end.
\item Provides a means to allow users to use almost any cite command via custom command
wrappers and the ability to manually disable all automatic internal hooks.
\item Mciteplus compatible sorting bibstyles are possible via the use of a ``mcitetail''
field in the \BibTeX\ database entries. See \MYsecref{sec:sortingbibstyles} for details.
\end{itemize}


\subsection{File List}
The files in this package are as follows:
\begin{description}
\item[\ttfamily mciteplus.sty] The mciteplus \LaTeX\ package.
\item[\ttfamily mciteplus\_doc.pdf] The user manual (this document).
\item[\ttfamily mciteplus\_doc.tex] The \LaTeX\ source of the user manual.
\item[\ttfamily mciteplus\_code.txt] Selected .bst modification and other code listings.
\item[\ttfamily apsrevM.bst] An mciteplus compatible version of REV\TeX4's apsrev.bst \cite{ctan:revtex}.
\item[\ttfamily apsrmpM.bst] An mciteplus compatible version of REV\TeX4's apsrmp.bst \cite{ctan:revtex}.
\textbf{Note: This is a sorting style. For it to work properly, you must set the
``mcitetail'' \BibTeX\ database field for every tail entry as mentioned in
\MYsecref{sec:sortingbibstyles}.}
\item[\ttfamily IEEEtranM.bst] An mciteplus compatible version of IEEEtran.bst \cite{ctan:ieeetranbst}.
\item[\ttfamily IEEEtranMN.bst] An mciteplus compatible version of IEEEtranN.bst (natbib compatible) \cite{ctan:ieeetranbst}.

\end{description}



\section{Basic Usage}


\subsection{Bibstyle (.bst) Requirements}
Mciteplus does require an mciteplus compatible bibstyle (.bst file).
\textbf{Unfortunately, bibstyles designed for use with mcite will not
work with mciteplus.} See \MYsecref{sec:bibstylemods} for
information on how to convert existing bibstyles for use with mciteplus.


\subsection{Package Loading and Options}
Mciteplus is invoked in the standard \LaTeX\ way:
\begin{MYlistblk}
\MYlongcmd
\usepackage[|itshape+options|upshape]{mciteplus}|end
\end{MYlistblk}
{\bfseries As mciteplus installs command wrappers over existing \MYshortcmd\cite|end\ commands,
it should be loaded last, even after other packages that normally are loaded last such
as hyperref \cite{ctan:hyperref}.}

Be aware that LaTeX may have to be run twice in order for the correct label
width to be calculated and used after entries are collapsed. When this happens,
mciteplus will issue the warning
``Rerun to ensure correct mciteplus label max width/count''.

\subsubsection{Package Options}
\label{sec:packageoptions}
Valid options are:

\indent\textbf{chapterbibrootbib}\MYemdash
This option must be specified when using the rootbib option/mode of the chapterbib package
\cite{ctan:chapterbib}. Mciteplus is not able to auto-detect the rootbib mode
of chapterbib directly because when invoking that mode, the user must run \LaTeX\ a second time
without the rootbib option. For all of these \LaTeX\ runs, keep this mciteplus option
enabled (i.e., as long as there is, or is going to be, a jobname.bbl file under
chapterbib). Beware of the possibility of an mciteplus status (i.e., head or tail)
conflict in an overall bibliography if the same entry is cited differently in
different parts of the document. See \MYsecref{sec:chapterbib} for more details on
the use of chapterbib.

\indent\textbf{debug}\MYemdash
Invoking this option will cause mciteplus to issue debugging information 
to the console whenever it processes: (1) a citation list; (2) an mciteplus aware
bibliography;  and (3) entries within the mciteplus bibliography. This may be helpful to
diagnose problems or to learn what tracking IDs, etc., mciteplus is using.

\indent\textbf{nohooks}\MYemdash
This option will prevent mciteplus from automatically hooking into \LaTeX's
\MYshortcmd\cite|end\ internals and/or autodetecting and interfacing with
external packages. The intended use is for advanced users who want to create
their own mciteplus' \MYshortcmd\cite|end\ wrappers. When using this option,
users will then have to manually define their own cite wrappers using the internal
mciteplus engine (which is always available to users regardless of whether this
option is selected or not). See \MYsecref{sec:customwrappers} for how to do this.
Beware that this option can cause some packages to issue errors when used with
mciteplus, especially those that alter the bibliography environments (e.g.,
citeref.sty, pageref.sty, etc.)


\subsection{Cite Format}
In the example given at the start of the introduction, the Glashow entry is
known as the ``head'' entry and those of Salam and
Weinberg are known as ``tails''. As with mcite, tails are declared within
the \MYshortcmd\cite|end\ command immediately after their head entry by prefixing them with a *:
\begin{MYlistblk}
\MYlongcmd
\cite{Glashow,*Salam,*Weinberg}|end
\end{MYlistblk}
Head entries may or may not have tails. However, every tail must have a head entry.
The tail entries for a group must all be declared (without duplication) when their
head entry is first cited. After that, the head entry may be recited as often as desired. 
It is possible, though bad practice, to redeclare the tails (or a subset thereof) when
reciting the head. This allows for the case of overall bibliographies in which the
same citation group is defined in multiple local bibliographies, all the entries
of which are later combined into a single overall bibliography.
However, new tails cannot be later added to an existing citation group.

Thus, each of the following lines will generate an error as they are \textbf{invalid}:
\begin{MYlistblk}
\MYlongcmd
\cite{*Salam} |% missing head declaration |\
| |\
\cite{Glashow,*Salam} \cite{Glashow,*Salam,*Weinberg} |% tails added after initial definition|\
| |\
\cite{Glashow,*Salam,*Salam} |% duplicate tails in initial definition|end
\end{MYlistblk}
Each of the following lines is \textbf{valid}:
\begin{MYlistblk}
\MYlongcmd
\cite{Glashow,*Salam,*Weinberg,Smith,*Jones} |% multiple groups OK |\
| |\
\cite{Glashow,*Salam,*Weinberg} \cite{Glashow} |% head can be recited anytime |\
| |\
\cite{Glashow,*Salam,*Weinberg} \cite{Glashow,*Salam,*Weinberg} |% restate previous definition |\
| |\
\cite{Glashow,*Salam,*Weinberg} \cite{Glashow,*Weinberg} |% restate part of previous definition |\
| |\
\cite[page 580 of Glashow]{Glashow,*Salam} |% optional argument is supported |\
| |\
\nocite{*} |% BibTeX wildcard OK, but use with caution |end
\end{MYlistblk}
The \BibTeX\ ``wildcard'' entry ``*'' for \MYshortcmd\nocite|end\ is allowed. However,
it is important to ensure that the entries \BibTeX\ will automatically add to the
bibliography from its databases will not get ``in between'' any head/tail groupings. This
will not be a problem with unsorted bibstyles as long as the head/tail groups are
cited prior to the invocation of the wildcard entry or as long as the head/tail group
entries are listed together and in the correct order in the \BibTeX\ databases. However, with
sorting bibstyles extra care must be taken to keep the head/tail groups together in
spite of the reordering process. See \MYsecref{sec:sortingbibstyles} for more information.

By default, mciteplus will issue an error message if it encounters
bibliography\footnote{Well, only those in a \texttt{mcitethebibliography} environment.}
entries that it has no record of. This will almost certainly be the case if the
\BibTeX\ wildcard entry is used. To disable the error messages and allow
mciteplus to automatically assume that all unknown entries are heads, just issue 
the \TeX\ conditional:
\begin{MYlistblk}
\MYlongcmd
\mciteErrorOnUnknownfalse|end
\end{MYlistblk}
before the bibliography it is to affect. Mciteplus will issue a warning message
to the console as a reminder if it detects the use of the \BibTeX\ wildcard entry.


\subsection{Punctuation Controls}
There are three kinds of punctuation/spacing mciteplus uses in the formatting of entries
in the bibliography: (1) Middle punctuation which is used between entries within a
collapsed group (typically ``; ''); (2) End punctuation which is used at the end
of each entry group (typically ``.''); and (3) separation punctuation/spacing
which is inserted before each head entry after the very first. Separation
punctuation/spacing is usually not used (it is typically ``\MYshortcmd\relax|end''),
but it is provided for because some bibliography styles may require unusual spacing
or a \MYshortcmd\par|end\  between the entries, but under mciteplus, bibstyles cannot
directly insert such things as \BibTeX\ does not know at run time which
entries are heads and which are tails.

Mciteplus defines the package defaults of each as:\footnote{{\ttfamily\ZAXbsl providecommand} is
used so that class files can declare their own mciteplus defaults.}
\begin{MYlistblk}
\MYlongcmd
\providecommand{\mcitedefaultmidpunct}{;\space}|\
\providecommand{\mcitedefaultendpunct}{.}|\
\providecommand{\mcitedefaultseppunct}{\relax}|end
\end{MYlistblk}
However, both the bibstyle and the user can override the defaults with the user having
the final say in the matter.

Bibstyles can override the package defaults via issuing the command:
\begin{MYlistblk}
\MYlongcmd
\mciteSetBstMidEndSepPunct{|bgroup|itshape+middle punctuation|egroup}{|bgroup|itshape+end punctuation|egroup}{|bgroup|itshape+separation punctuation|egroup}|end
\end{MYlistblk}
\MYshortcmd\mciteSetBstMidEndSepPunct|end\ must be issued at the end of the entry it is to
affect (prior to the next \MYshortcmd\bibitem|end) and the values given will remain in effect
until specified again.

The user can override the bibstyle by issuing the command:
\begin{MYlistblk}
\MYlongcmd
\mciteSetMidEndSepPunct{|bgroup|itshape+middle punctuation|egroup}{|bgroup|itshape+end punctuation|egroup}{|bgroup|itshape+separation punctuation|egroup}|end
\end{MYlistblk}
before the bibliography it is to affect. However, this poses a problem because
the bibstyle may or may not specify end punctuation for each entry (e.g., such
as when an entry ends with a URL or an abbreviated journal name). To provide for
this, mciteplus provides a \TeX\ conditional \MYshortcmd\ifmciteBstWouldAddEndPunct|end,
that a bibstyle can set to true if and only if \BibTeX\ would add end punctuation
to the given entry. (See \MYsecref{sec:bibstylemods} for how bibstyle designers
should implement this feature.) If the bibstyle provides this feature, and it is
recommended that they do, then users can take advantage of this conditional in their use
of \MYshortcmd\mciteSetMidEndSepPunct|end:
\begin{MYlistblk}
\MYlongcmd
\mciteSetMidEndSepPunct{;\space}{\ifmciteBstWouldAddEndPunct.\else\fi}{\relax}|end
\end{MYlistblk}
and thus still avoid the ``double period'' problem under their
own custom punctuation.

After issuing a user defined \MYshortcmd\mciteSetMidEndSepPunct|end\ for a bibliography,
a user can restore control back to the bibstyle for later bibliographies by issuing:
\begin{MYlistblk}
\MYlongcmd
\mciteSetMidEndSepPunct{\mcitebstmidpunct}{\mcitebstendpunct}{\mcitebstseppunct}|end
\end{MYlistblk}
before the bibliography(ies) to be affected.



\subsection{Counters}
\label{sec:counters}
Mciteplus provides two \LaTeX\ counters, \MYshortcmd+mcitebibitemcount|end\ and
\MYshortcmd+mcitesubitemcount|end, which track the number of head, and tail entries
for each head, respectively. At each head entry, \MYshortcmd+mcitebibitemcount|end\ is
incremented and \MYshortcmd+mcitesubitemcount|end\ is reset, by/to one.
At each tail entry (or sublisted head entry under sublist modes ``b'', ``f'' and ``h''
as described in \MYsecref{sec:sublistlabelbeginend}), \MYshortcmd+mcitesubitemcount|end\ 
is incremented by one after each subitem label is rendered.
Be aware that \MYshortcmd+mcitebibitemcount|end\ is \emph{not} what is used to
actually generate the entry labels in the bibliography as \LaTeX\ handles this
the same way it normally does (typically using it's own counter, 
\MYsmalltt{enumiv}).

These counters can be referenced by the user to generate sublist labels
(\MYsecref{sec:sublistlabelbeginend}), to determine maximum label
widths (\MYsecref{sec:maxwidths}) as well as to control
other counters (e.g., \MYsecref{sec:mcitethebibliographyhooks}).


\subsubsection{Bibitem Counter Reset}
\label{sec:bibitemcounterreset}
By default, mciteplus resets \MYshortcmd+mcitebibitemcount|end\ at the
start of each bibliography. However, in some documents with multiple bibliographies,
the entries are to be numbered consecutively throughout the document and, thus, 
the entry count should not reset at the start of each bibliography. For
such cases, you can issue a:
\begin{MYlistblk}
\MYlongcmd
\mciteResetBibitemCountfalse|end
\end{MYlistblk}
before a bibliography to disable the counter reset (so that the maximum
label widths and counts will be correct, see \MYsecref{sec:labelwidths}).
There is also a \MYshortcmd\mciteResetBibitemCounttrue|end\ which reenables the
resetting of the counter.




\subsection{Sublists}
\label{sec:sublists}
Sublisting within entry groups is supported by mciteplus. For example:
\begin{MYapsrevthebibliography}{1}
\bibitem[{\citenamefont{Glashow}(1961)}]{Glashow}
\bibinfo{author}{\bibfnamefont{S.~L.} \bibnamefont{Glashow}},
  \bibinfo{journal}{Nucl. Phys.} \textbf{\bibinfo{volume}{22}},
  \bibinfo{pages}{579} (\bibinfo{year}{1961});
a) \bibinfo{author}{\bibfnamefont{A.}~\bibnamefont{Salam}}, in
  \emph{\bibinfo{booktitle}{Elementary Particle Theory}}, edited by
  \bibinfo{editor}{\bibfnamefont{N.}~\bibnamefont{Svartholm}}
  (\bibinfo{publisher}{Almquist and Wiksell}, \bibinfo{address}{Stockholm},
  \bibinfo{year}{1968}), pp. \bibinfo{pages}{367--377};
b) \bibinfo{author}{\bibfnamefont{S.}~\bibnamefont{Weinberg}},
  \bibinfo{journal}{Phys. Rev. Lett.} \textbf{\bibinfo{volume}{19}},
  \bibinfo{pages}{1264} (\bibinfo{year}{1967}).
\end{MYapsrevthebibliography}
The sublist labels (e.g., ``a)'', ``b)'', etc.) are based on the mciteplus
provided \LaTeX\ counter,
\MYshortcmd+mcitesubitemcount|end, which is reset at the start
of each head.

\subsubsection{Sublist Modes}
\label{sec:sublistmodes}
The standard sublist mode ``s'' (as show above) is to
begin numbering with the first tail.

Mciteplus can begin numbering with the head entries, but
omit the sublabel for the heads via sublist mode ``b'':
\begin{MYapsrevthebibliography}{1}
\bibitem[{\citenamefont{Glashow}(1961)}]{Glashow}
\bibinfo{author}{\bibfnamefont{S.~L.} \bibnamefont{Glashow}},
  \bibinfo{journal}{Nucl. Phys.} \textbf{\bibinfo{volume}{22}},
  \bibinfo{pages}{579} (\bibinfo{year}{1961});
b) \bibinfo{author}{\bibfnamefont{A.}~\bibnamefont{Salam}}, in
  \emph{\bibinfo{booktitle}{Elementary Particle Theory}}, edited by
  \bibinfo{editor}{\bibfnamefont{N.}~\bibnamefont{Svartholm}}
  (\bibinfo{publisher}{Almquist and Wiksell}, \bibinfo{address}{Stockholm},
  \bibinfo{year}{1968}), pp. \bibinfo{pages}{367--377};
c) \bibinfo{author}{\bibfnamefont{S.}~\bibnamefont{Weinberg}},
  \bibinfo{journal}{Phys. Rev. Lett.} \textbf{\bibinfo{volume}{19}},
  \bibinfo{pages}{1264} (\bibinfo{year}{1967}).
\end{MYapsrevthebibliography}

To sublabel the first entries of each group (heads) use
the sublist mode ``f'': 
\begin{MYapsrevthebibliography}{1}
\bibitem[{\citenamefont{Glashow}(1961)}]{Glashow}
a) \bibinfo{author}{\bibfnamefont{S.~L.} \bibnamefont{Glashow}},
  \bibinfo{journal}{Nucl. Phys.} \textbf{\bibinfo{volume}{22}},
  \bibinfo{pages}{579} (\bibinfo{year}{1961});
b) \bibinfo{author}{\bibfnamefont{A.}~\bibnamefont{Salam}}, in
  \emph{\bibinfo{booktitle}{Elementary Particle Theory}}, edited by
  \bibinfo{editor}{\bibfnamefont{N.}~\bibnamefont{Svartholm}}
  (\bibinfo{publisher}{Almquist and Wiksell}, \bibinfo{address}{Stockholm},
  \bibinfo{year}{1968}), pp. \bibinfo{pages}{367--377};
c) \bibinfo{author}{\bibfnamefont{S.}~\bibnamefont{Weinberg}},
  \bibinfo{journal}{Phys. Rev. Lett.} \textbf{\bibinfo{volume}{19}},
  \bibinfo{pages}{1264} (\bibinfo{year}{1967}).
\end{MYapsrevthebibliography}
Entries without tails (single heads) are not sublisted under sublist
mode ``f''. To enable that, use sublist mode ``h''.

As with punctuation, the sublist mode can be specified by both the bibstyle
as well as the user with the latter overriding the former:
\begin{MYlistblk}
\MYlongcmd
\mciteSetBstSublistMode{|itshape+mode|upshape} |% for use by bibstyles|\
| |\
\mciteSetSublistMode{|itshape+mode|upshape} |% for use by the user in the document|end
\end{MYlistblk}
where \textit{mode} is one of:
\begin{enumerate}
\item[\textbf{d}] \MYemdash ``Default'', do not alter whatever mode is currently in effect.
For bibstyles, this is essentially a NOOP, but when invoked by the user, the bibstyle is
allowed to alter the sublist mode.
\item[\textbf{n}] \MYemdash No sublist, do not sublist the entries.
\item[\textbf{s}] \MYemdash Sublist mode ``s'', sublist the tail entries.
\item[\textbf{b}] \MYemdash Sublist mode ``b'', sublist the tail entries, start the count with their
head, but omit labeling the head.
\item[\textbf{f}] \MYemdash Sublist mode ``f'', sublist the tail entries including their head.
\item[\textbf{h}] \MYemdash Sublist mode ``h'', implies sublist mode ``f'', but sublist all entries,
even heads without tails.
\end{enumerate}
Unless specified otherwise, sublisting will not be done.

\subsubsection{Sublist Label, Begin and End}
\label{sec:sublistlabelbeginend}
Mciteplus allows bibstyles and users to specify the sublist label form used,
the code that gets executed at the start of the sublist and the code
that ends a sublist. Mciteplus defines the package defaults of each as:
\begin{MYlistblk}
\MYlongcmd
\providecommand{\mcitedefaultsublistlabel}{\alph{mcitesubitemcount})\space}|\
\providecommand{\mcitedefaultsublistbegin}{\relax}|\
\providecommand{\mcitedefaultsublistend}{\relax}|end
\end{MYlistblk}
The result of which is as shown in the previous examples.
However, both the bibstyle and the user can override the defaults with the user having
the final say in the matter. The sublist label/begin/end code can be specified by both
the bibstyle as well as the user with the latter overriding the former:
\begin{MYlistblk}
\MYlongcmd
\mciteSetBstSublistLabelBeginEnd{|itshape+label|upshape}{|itshape+begin|upshape}{|itshape+end|upshape} |% for use by bibstyles|\
| |\
\mciteSetSublistLabelBeginEnd{|itshape+label|upshape}{|itshape+begin|upshape}{|itshape+end|upshape} |% for use by the user in the document|end
\end{MYlistblk}
This flexibility provides a way to use customized environments for the sublists:
\begin{MYlistblk}
\MYlongcmd
\mciteSetSublistMode{s}|\
\mciteSetSublistLabelBeginEnd{\item[\arabic{mcitebibitemcount}.\alph{mcitesubitemcount})]}{\begin{enumerate}}{\end{enumerate}}|end
\end{MYlistblk}
which yields:
\begin{MYapsrevthebibliography}{1}
\bibitem[{\citenamefont{Glashow}(1961)}]{Glashow}
\bibinfo{author}{\bibfnamefont{S.~L.} \bibnamefont{Glashow}},
  \bibinfo{journal}{Nucl. Phys.} \textbf{\bibinfo{volume}{22}},
  \bibinfo{pages}{579} (\bibinfo{year}{1961});
\begin{enumerate}
\item[1.a)] \bibinfo{author}{\bibfnamefont{A.}~\bibnamefont{Salam}}, in
  \emph{\bibinfo{booktitle}{Elementary Particle Theory}}, edited by
  \bibinfo{editor}{\bibfnamefont{N.}~\bibnamefont{Svartholm}}
  (\bibinfo{publisher}{Almquist and Wiksell}, \bibinfo{address}{Stockholm},
  \bibinfo{year}{1968}), pp. \bibinfo{pages}{367--377};
\item[1.b)] \bibinfo{author}{\bibfnamefont{S.}~\bibnamefont{Weinberg}},
  \bibinfo{journal}{Phys. Rev. Lett.} \textbf{\bibinfo{volume}{19}},
  \bibinfo{pages}{1264} (\bibinfo{year}{1967}).
\end{enumerate}
\end{MYapsrevthebibliography}

After issuing a user defined \MYshortcmd\mciteSetSublistLabelBeginEnd|end\ for a bibliography,
a user can restore control back to the bibstyle for later bibliographies by issuing:
\begin{MYlistblk}
\MYlongcmd
\mciteSetSublistLabelBeginEnd{\mcitebstsublistlabel}{\mcitebstsublistbegin}{\mcitebstsublistend}|end
\end{MYlistblk}
before the bibliography(ies) to be affected.



\subsubsection{Referencing Sublist Labels}
\label{sec:refsublistlabel}
It is not possible to reference the tails using  \MYshortcmd\cite|end:
\begin{MYlistblk}
\MYlongcmd
\cite{Salam}  |% invalid, tail redeclared as a head|\
\cite{*Salam} |% invalid, tail cannot be referenced|end
\end{MYlistblk}
and even if it were allowed, it would likely cause problems with packages
that compress and sort citation numbers.

However, mciteplus does provide two commands that work much like \MYshortcmd\ref|end\ 
and \MYshortcmd\pageref|end\ that can be used to reference the sublabeled entries
and the page numbers they appear on:
\begin{MYlistblk}
\MYlongcmd
\mciteSubRef[|itshape+track ID|upshape]{|itshape+cite key|upshape}|\
\mciteSubPageRef[|itshape+track ID|upshape]{|itshape+cite key|upshape}|end
\end{MYlistblk}
where \textit{cite key} is a (single) citation key and \textit{track ID} is
the tracking ID (\MYsecref{sec:trackingandaux}) string which mciteplus uses
to uniquely identify bibliographies. Please note that these commands will not work
for entries that are not sublabeled (e.g., heads with tails if the sublist
mode is not ``f'' or ``h''). The tracking IDs used by each bibliography can be
displayed by loading mciteplus with the ``debug'' package option (\MYsecref{sec:packageoptions}).
For virtually all single bibliography documents, the default tracking ID should work
fine. However, when multiple bibliographies are in use, the tracking ID may
have to be specified so that mciteplus will know which bibliography entry to
reference. Some packages such as chapterbib, may even put the same citation
in multiple bibliographies.

For the part/chapter bibliographies of chapterbib.sty, there is no need to specify a
tracking ID as mciteplus will automatically use the correct one within each part.
The tracking ID of the duplicate bibliographies of chapterbib (produced by
chapterbib's ``duplicate'' package option) is given by ``chapterbib\textit{inputfile}.bbl'',
where ``\textit{inputfile}'' is the include file the given citation occurred in.
For the rootbib bibliography (produced by chapterbib's ``rootbib'' package option),
the tracking ID is of the form ``chapterbib\textit{jobname}'', where
``\textit{jobname}'' is the name of the main .tex document file (without the .tex
suffix).

For multibib.sty, the tracking ID is given by ``multibib\textit{secname}'', where
``\textit{secname}'' is the multibib bibliography section name (as declared
via \MYshortcmd\newcites|end).

For multibbl.sty, the tracking ID is given by ``multibbl\textit{secname}'', where
``\textit{secname}'' is the multibbl bibliography section name.

For bibunits.sty, mciteplus will automatically use the correct tracking ID within
each bibunit, but to reference an entry outside of a bibunit, the tracking ID of
each bibunit is given by ``bibunits\textit{unitname}'', where ``\textit{unitname}''
is the name of the auxiliary file of the desired bibunit without the .aux suffix
(e.g., ``bu1'', ``bu2'', etc.).

The form of the replacement text of \MYshortcmd\mciteSubRef|end\ is defined by
mciteplus as:
\begin{MYlistblk}
\MYlongcmd
\providecommand{\mcitesubrefform}{\arabic{mcitebibitemcount}.\alph{mcitesubitemcount}}|end
\end{MYlistblk}
which can be redefined (via \MYshortcmd\renewcommand|end) by the user as desired.


\subsubsection{Bibitem Argument Macros}
\label{sec:bibitemarguments}
When creating sublist labels (or the maximum width forms discussed in 
\MYsecref{sec:maxwidths}), the user may wish to have access to the
argument(s) of the \MYshortcmd\bibitem|end\ of the entries. Mciteplus
provides these as macros. The required argument of the current
\MYshortcmd\bibitem|end, which is usually the citation key, is provided
as \MYshortcmd\mciteBibitemArgI|end. The optional argument of the current
\MYshortcmd\bibitem|end, which, if present, is provided as 
\MYshortcmd\mciteBibitemOptArgI|end. There is also the \TeX\ conditional
\MYshortcmd\ifmciteBibitemOptArgI|end, which will evaluate true if the
optional argument to \MYshortcmd\bibitem|end\ is present. 

Lastly, there are \MYshortcmd\mciteCurheadBibitemArgI|end, 
\MYshortcmd\mciteCurheadBibitemOptArgI|end\ and
\MYshortcmd\ifmciteCurheadBibitemOptArgI|end\ which contain the
same argument information, but for the head of the current entry
group.



\subsection{Label Widths}
\label{sec:labelwidths}

\textbf{Note: Problems with incorrect label widths will usually manifest themselves
as incorrect label spacing and/or bibliography entry text that is not properly
aligned, but often will not otherwise generate warning messages or errors.}


\subsubsection{The Bibliography Sample Label}
\label{sec:bibsamplelabel}
With most bibstyles, \BibTeX\ produces what is known as a ``sample label,''
which is the widest label used by the bibliography entries. This sample label is
recorded as the argument to \MYshortcmd\thebibliography|end\ when \BibTeX\ creates
the .bbl bibliography file so as to provide a way for the \MYshortcmd\thebibliography|end\ 
environment to know in advance how much space to reserve for the labels.

For example, \BibTeX\ may produce something like this in the .bbl file for a
bibliography with ten entries:\footnote{Some bibstyles use a number equal to the
number of entries, others use a form such as ``1X'', where ``X'' is a number of zeros
such that the sample label length matches that of the number of entries (e.g.,
``10'' if there are 10--99 entries). With most fonts, the numerical digits
all have the same width so ``10'' has the same width as ``99''.}
\begin{MYlistblk}
\MYlongcmd
\begin{thebibliography}{10}|end
\end{MYlistblk}
Mciteplus provides users access to the original \BibTeX-produced sample label
as the macro \MYshortcmd\mciteorgbibsamplelabel|end.

A problem arises with the sample label system when mciteplus combines multiple entries in
that the \BibTeX\ generated sample label in the .bbl file will not be
correct after the tail entries are no longer counted as actual entries. With the
original mcite package, this would often result in incorrect label spacing. Mciteplus
addresses this problem by actively tracking the maximum width label used in each bibliography.

The sample label form that mciteplus forwards to \MYshortcmd+thebibliography|end\ 
is defined as:
\begin{MYlistblk}
\MYlongcmd
\def\mcitebibsamplelabel{\rule{\mcitemaxwidthbibitem sp}{0.2pt}}|end
\end{MYlistblk}
which is simply a rule with a width equal to the widest label as determined
by the maximum width system discussed in \MYsecref{sec:maxwidths}.
If necessary, this can be redefined by the user via \MYshortcmd\renewcommand|end.

Most class files will not be bothered by the fact that a rule rather than
actual digits is being used for the sample label. However, if this becomes
a problem, something like:
\begin{MYlistblk}
\MYlongcmd
\renewcommand{\mcitebibsamplelabel}{\mcitemaxcountbibitem}|end
\end{MYlistblk}
can be tried where \MYshortcmd\mcitemaxcountbibitem|end\ will contain
the number of the last entry in the current bibliography as discussed
in \MYsecref{sec:maxcounts}.

The \MYshortcmd+thebibliography|end\ environments of some \LaTeX\ classes
(e.g., REV\TeX) perform their own label width calculations and ignore the sample
label entirely making the issue a moot point. However, the most common \LaTeX\ 
classes do depend on the sample label being accurate.


\subsubsection{Maximum Widths}
\label{sec:maxwidths}
The problem of knowing the widest of a group of labels in advance of their 
actual rendering\footnote{Obviously, solving this problem requires more than one
\LaTeX\ pass as well as the use of the auxiliary file to store information about the
widest label which will be read during the second pass.} is commonly encountered
in \LaTeX\ (e.g., lists, section numbering within the table of contents, etc.) and
is an issue that, in this author's opinion, has not been dealt with adequately by
the base \LaTeX\ system.\footnote{Packages such as Scott Pakin's eqparbox.sty
\cite{ctan:eqparbox} are most welcome solutions to this type of problem.} With
regard to mciteplus bibliographies, not only is the width of the head entries (i.e.,
the sample label) an issue, but potentially so are the sublist (tail) labels if
they are rendered in enumerated list form rather than inline.

To assist the user in determining the required width of labels, mciteplus provides
its own maximum width and count tracking facilities. Both the bibitem (main entry) and
subitem (sublist entry) label widths can tracked independently of each other.

In order to measure widths, mciteplus must know what exactly it is to measure.
This is specified by macros that contain the ``width forms'' which are simply
``text'' whose width will be measured for each bibitem or subitem. The package
defaults of each are defined as:
\begin{MYlistblk}
\MYlongcmd
\providecommand{\mcitedefaultmaxwidthbibitemform}{\arabic{mcitebibitemcount}}|\
\providecommand{\mcitedefaultmaxwidthsubitemform}{\alph{mcitesubitemcount})}|end
\end{MYlistblk}
There are also the initial width forms that will be used only during the very first
\LaTeX\ run before the maximum label widths are actually known (i.e., when the
auxiliary file is absent):
\begin{MYlistblk}
\MYlongcmd
\providecommand{\mcitedefaultmaxwidthbibitemforminit}{\mciteorgbibsamplelabel}|\
\providecommand{\mcitedefaultmaxwidthsubitemforminit}{a)}|end
\end{MYlistblk}
The initial width forms will be evaluated with the bibitem and subitem counters both set to one.
The purpose of the initial width forms is just to provide a rough guess as to the
label widths to try to avoid large formatting changes or errors within label
formatting code that cannot deal with unknown, or zero, widths.

As with the punctuation and sublist controls, both the bibstyle and the user can 
specify the width forms with the latter overriding the former:
\begin{MYlistblk}
\MYlongcmd
\mciteSetBstMaxWidthForm[|itshape+init|upshape]{|itshape+type|upshape}{|itshape+form|upshape} |% for use by bibstyles|\
| |\
\mciteSetMaxWidthForm[|itshape+init|upshape]{|itshape+type|upshape}{|itshape+form|upshape} |% for use by the user in the document|end
\end{MYlistblk}
where \textit{type} is ``\texttt{bibitem}'' or ``\texttt{subitem}''. The optional argument 
\textit{init} is used to specify the initial width form that will be used. If it is not specified,
the current initial width form in use will not be altered. The code specified in the forms
should be self contained (i.e., not have commands which require an external environment
such as \MYshortcmd+\item|end) as well as allow reevaluation at any time (e.g., should
not do things such as alter counters). Note that these restrictions
do not apply to the actual sublist code (\MYsecref{sec:sublistlabelbeginend}) and is
one reason why mciteplus distinguishes between that which measures the width of the 
a label (discussed in this section) and that which actually forms the label 
(\MYsecref{sec:sublistlabelbeginend}). If desired, the current width forms (see below)
can be used within the actual sublist code.

After issuing a user defined \MYshortcmd\mciteSetMaxWidthForm|end\ for a bibliography,
a user can restore control back to the bibstyle for later bibliographies by issuing:
\begin{MYlistblk}
\MYlongcmd
\mciteSetMaxWidthForm[\mcitebstmaxwidthbibitemforminit]{bibitem}{\mcitebstmaxwidthbibitemform}|\
\mciteSetMaxWidthForm[\mcitebstmaxwidthsubitemforminit]{subitem}{\mcitebstmaxwidthsubitemform}|end
\end{MYlistblk}
before the bibliography(ies) to be affected.

Please note that for bibstyles, there
is no use in specifying an initial width form via \MYshortcmd+\mciteSetBstMaxWidthForm[]|end\ 
after the bibliography begins because, by that point, it is too late as the sample label
has already been forwarded. However, this is not a problem because by default, the initial
width form is the original sample label as given by the bibstyle.\footnote{Another
possible approach would be to specify the bibitem width form before the bibliography begins,
possibly surrounding the entire environment in a group to keep the changes local.}
This is not an issue for the normal bibitem width form or the subitem width forms as
they are not evaluated until the first \MYshortcmd\bibitem|end.

The user has access to the currently used bibitem and subitem width forms and
their maximum widths as the macros:
\begin{MYlistblk}
\MYlongcmd
\mcitemaxwidthbibitemform |% current bibitem width form |\
\mcitemaxwidthsubitemform |% current subitem width form |\
\mcitemaxwidthbibitem |% the maximum width of the bibitem form|\
\mcitemaxwidthsubitem |% the maximum width of the subitem form|end
\end{MYlistblk}
The maximum widths are macros containing an integer which is the width
in \TeX\ scaled points (sp).\footnote{In \TeX, a scaled point is the smallest
dimensional unit available and is equal to $\frac{1}{65536}$ of a point.} The
widths are not length commands (e.g., a dimension or skip register), but their
values can be assigned to a length command by appending a ``sp'' to the
length specification:
\begin{MYlistblk}
\MYlongcmd
\newlength{\MYlength}|\
\setlength{\MYlength}{\mcitemaxwidthsubitem sp}|end
\end{MYlistblk}


\subsubsection{Maximum Counts}
\label{sec:maxcounts}
Also available are the macros:
\begin{MYlistblk}
\MYlongcmd
\mcitemaxcountbibitem|\
\mcitemaxcountsubitem|end
\end{MYlistblk}
which contain the value of the highest count of the bibitem or subitem\footnote{For
subitems, it is the highest count that occurred \textit{during} the generation of the
sublist labels. Because the subitem counter is advanced by one \textit{after} each
sublist label is rendered, the actual counter itself does reach one count higher than
its maxcount, but that value was not actually used.} 
counters, respectively, in the current bibliography. The maxcount
macros are not counters, but they can be used to set them, among other things.
There are no ``forms'' associated with the maxcounts and their initial values
(during the very first \LaTeX\ run in the absence of an auxiliary file) will be zero.
So, any code that references them should be able to handle (without error) the case
of zero.

If, at the end of the bibliography, the correct maximum widths and counts
do not match what mciteplus used at the start of the run, mciteplus will issue
a package warning to request that \LaTeX\ be rerun.


\subsubsection{Non-Numerical Labels}
\label{sec:nonnumericallabels}
The maxwidth forms may have to be changed for some non-numerical bibstyles.
For example, under alpha.bst \cite{ctan:alphabst}, the entry labels are
carried within the optional arguments of the \MYshortcmd+bibitem|end, in
which case
\begin{MYlistblk}
\MYlongcmd
\mciteSetMaxWidthForm{bibitem}{\mciteBibitemOptArgI}|end
\end{MYlistblk}
would be appropriate when using such a bibstyle under mciteplus. Actually,
it is better to do such setup within the bibliography code of the
bibstyle (.bst) so that the maxwidth form will be correct by default
without any effort on the part of the user:
\begin{MYlistblk}
\MYlongcmd
\mciteSetBstMaxWidthForm{bibitem}{\mciteBibitemOptArgI}|end
\end{MYlistblk}
This should not be an issue with natbib compatible bibstyles, even
under the author-date mode, as natbib automatically does its own label
generation and management.



\subsection{Example of Use}
\label{sec:exampleofuse}
Perhaps the maxwidth system is easiest to understand from an example:
\begin{MYlistblk}
\MYlongcmd
\mciteSetSublistMode{f}|% start numbering with the first item|\
\mciteSetMaxWidthForm{subitem}{\scriptsize\textbf{\roman{mcitesubitemcount})}}|% small, bold, roman|\
| |\
\newcommand{\MYlistsetup}{\relax|\
\setlength{\labelwidth}{\mcitemaxwidthsubitem sp}|% reserve room for widest label|\
\setlength{\labelsep}{5pt}|% 5pt label/text separation|\
\setlength{\itemindent}{0pt}% no indent|\
\setlength{\leftmargin}{\labelwidth}|% text to left of label|\
\addtolength{\leftmargin}{\labelsep}}|% and separation space|\
| |\
\mciteSetSublistLabelBeginEnd{\item}{\begin{list}{\hfill\mcitemaxwidthsubitemform}{\MYlistsetup}}{\end{list}}|end
\end{MYlistblk}
which produces a sublist via a customized list with small, bold and right aligned
roman numerals as labels and with block indented text. Note that in the list setup
code, we need to know the amount of room to reserve for the labels. Mciteplus provides this as
\MYshortcmd\mcitemaxwidthsubitem|end\ (to which the units ``sp'' must be appended when
assigning it to a length command). The maxwidth form itself,
\MYshortcmd\mcitemaxwidthsubitemform|end\ was used in the specification of the
labels within the list setup, but its definition (as used within
\MYshortcmd\mciteSetMaxWidthForm|end) could have been respecified in the list
setup code if desired. Once the list is setup, each sublabel is generated by a
simple call to  \MYshortcmd\item|end. So, this example shows a case in which the code
actually used to produced the sublabels at run time (\MYshortcmd\item|end)
is not (directly) the same as that code which is used to measure the sublabel
widths. Indeed, as it is not self contained, \MYshortcmd\item|end\ cannot
even be used in the definition of a maxwidth form.




\section{Bibstyle (.bst) Modification}
\label{sec:bibstylemods}
Original mcite bibstyles cannot be used with mciteplus. They will have to be remodified
as described below.

To modify a standard bibstyle (.bst) for use with mciteplus (unsrt.bst
is used as an example here), find the instances of \MYshortcmd\begin{thebibliography}|end\ 
and \MYshortcmd\end{thebibliography}|end\ in the \MYshortcmd+begin.bib|end\ 
and \MYshortcmd+end.bib|end\ functions and change them to use
\MYshortcmd+mcitethebibliography|end. It may also be a good idea to test for
the existence of \MYshortcmd+mcitethebibliography|end\ in case the user ever
forgets to load mciteplus.sty.

{\bfseries Note: Many of the functions,
or parts of functions, shown below are listed in the included file \MYsmalltt{mciteplus\_code.txt},
which can be used to cut and paste from. Beware of slight code differences between the different bibstyles.
For example, some use \MYshortcmd+longest.label|end\ like this example, but others may
use something else like \MYshortcmd+number.label| int.to.str$|end\ and/or may have extra code
after the \MYshortcmd\begin{mcitethebibliography}|end\ line. Leave these other things as they are.}

Thus, 
\begin{MYlistblk}
\MYlongcmd
FUNCTION {begin.bib}|\
{ preamble$ empty$|\
    'skip$|\
    { preamble$ write$ newline$ }|\
  if$|\
  "\begin{thebibliography}{"  longest.label  * "}" * write$ newline$|\
}|\
| |\
.|\
.|\
| |\
FUNCTION {end.bib}|\
{ newline$|\
  "\end{thebibliography}" write$ newline$|\
}|end
\end{MYlistblk}
becomes (change the string "unsrtM.bst" to the name of your new .bst file):
\begin{MYlistblk}
\MYlongcmd
FUNCTION {begin.bib}|\
{ preamble$ empty$|\
    'skip$|\
    { preamble$ write$ newline$ }|\
  if$|\
  "\ifx\mcitethebibliography\mciteundefinedmacro"|\
  write$ newline$|\
  "\PackageError{unsrtM.bst}{mciteplus.sty has not been loaded}"|\
  write$ newline$|\
  "{This bibstyle requires the use of the mciteplus package.}\fi"|\
  write$ newline$
  "\begin{mcitethebibliography}{"  longest.label  * "}" * write$ newline$|\
}|\
| |\
.|\
.|\
| |\
FUNCTION {end.bib}|\
{ newline$|\
  "\end{mcitethebibliography}" write$ newline$|\
}|end
\end{MYlistblk}

You may also want to declare the sublist mode via a
\MYshortcmd\mciteSetBstSublistMode|end, as mentioned in \MYsecref{sec:sublists},  
just after the \MYshortcmd+mcitethebibliography|end\ environment
begins, that will be used if the user does not otherwise specify a sublist mode,
as well as other bibliography setup commands.

For example:
\begin{MYlistblk}
\MYlongcmd
FUNCTION {begin.bib}|\
{ preamble$ empty$|\
    'skip$|\
    { preamble$ write$ newline$ }|\
  if$|\
  "\ifx\mcitethebibliography\mciteundefinedmacro"|\
  write$ newline$|\
  "\PackageError{unsrtM.bst}{mciteplus.sty has not been loaded}"|\
  write$ newline$|\
  "{This bibstyle requires the use of the mciteplus package.}\fi"|\
  write$ newline$|\
  "\begin{mcitethebibliography}{"  longest.label  * "}" * write$ newline$|\
  "\mciteSetBstSublistMode{b}"|\
  write$ newline$|\
  "\mciteSetBstMaxWidthForm{subitem}{\alph{mcitesubitemcount})}"|\
  write$ newline$|\
  "\mciteSetBstSublistLabelBeginEnd{\mcitemaxwidthsubitemform\space}"|\
  write$ newline$|\
  "{\relax}{\relax}"|\
  write$ newline$|\
}|end
\end{MYlistblk}
will specify a \MYshortcmd+mcitethebibliography|end\ that defaults to using the sublist
``b'' mode and enumerates the subentries using letters. The middle and end punctuation
was not be specified via \MYshortcmd\mciteSetBstMidEndSepPunct|end\ and so the package
defaults will be used. In this example, the subitem maxwidth form is not so
important as the labels are rendered inline, but other types of lists may require it.

Next, find the \MYshortcmd+fin.entry|end\ function:
\begin{MYlistblk}
\MYlongcmd
FUNCTION {fin.entry}|\
{ add.period$|\
  write$|\
  newline$|\
}|end
\end{MYlistblk}
and change it to (which you can cut and paste from \MYsmalltt{mciteplus\_code.txt}):
\begin{MYlistblk}
\MYlongcmd
|% mciteplus fin.entry|\
|%|\
|% pushes true (1), if add.period$ would add a period to the string on the stack|\
|% pushes false (0), otherwise|\
|% Uses text.length$ to avoid full string comparison and two copies of string.|\
|% Requires one copy of string on stack.|\
INTEGERS {would.add.period.textlen}|\
FUNCTION {would.add.period}|\
{ duplicate$|\
  add.period$|\
  text.length$|\
  'would.add.period.textlen :=|\
  duplicate$|\
  text.length$|\
  would.add.period.textlen =|\
    { #0 }|\
    { #1 }|\
  if$|\
}|\
| |\
FUNCTION {fin.entry}|\
{ would.add.period|\
    { "\relax" * write$ newline$|\
      "\mciteBstWouldAddEndPuncttrue" write$ newline$|\
      "\mciteSetBstMidEndSepPunct{\mcitedefaultmidpunct}"|\
      write$ newline$|\
      "{\mcitedefaultendpunct}{\mcitedefaultseppunct}\relax"|\
    }|\
    { "\relax" * write$ newline$|\
      "\mciteBstWouldAddEndPunctfalse" write$ newline$|\
      "\mciteSetBstMidEndSepPunct{\mcitedefaultmidpunct}"|\
      write$ newline$|\
      "{}{\mcitedefaultseppunct}\relax"|\
    }|\
  if$|\
  write$|\
  newline$|\
  "\EndOfBibitem" write$|\
}|\
|% end mciteplus fin.entry|end
\end{MYlistblk}

The \MYshortcmd\mciteSetBstMidEndSepPunct|end\ line is broken into two lines (after the first argument)
to prevent \BibTeX\ from breaking what would be a long single line (over 80
columns) at an unacceptable place.

By providing a way for the .bst to tell mciteplus not to use an end period
if \BibTeX's \MYshortcmd+add.period$|end\ command would not, we avoid the double end period
problem. The \MYshortcmd\mciteBstWouldAddEndPuncttrue/false|end\ flag does not in itself
do anything. However, if set appropriately by the .bst, it can be employed
by the user in the definition custom punctuation via \MYshortcmd\mciteSetMidEndSepPunct|end.

Other .bst files may hardcode values other than the defaults and/or use
more complex conditions as is needed for that particular bibstyle.

The \MYshortcmd\EndOfBibitem|end\ is defined by mciteplus.sty as a macro containing
\MYshortcmd\relax|end. Although it does nothing at present, it's purpose is to make it
easier to identify the end of each bibliography entry for the purposes of inspection
or parsing.


\subsection{Sorting Bibstyles}
\label{sec:sortingbibstyles}
Perhaps surprisingly, it is possible to use mciteplus with sorting bibstyles.
The difficulty is that \BibTeX\ must keep an entry group together, in citation
order with the tails immediately following their respective head, despite
sorting the entries. This is possible with the use of special bibstyle code
if the user is willing and able to declare all the tails via a special entry
field ``mcitetail'', which should be set to ``yes'' for each tail, in the
\BibTeX\ database.\footnote{Thus, we have to specify the tails
twice\MYemdash once for \LaTeX\ and once for \BibTeX\ because there is no
other way to get the information to \BibTeX. This limitation
helped to inspire \LaTeX-based bibliography processing as is done with the
amsrefs \cite{ctan:amsrefs} and biblatex \cite{ctan:biblatex} packages.} Entries without this
field, or that have it set to ``no'', will be sorted normally (as head entries).
Of course, bibstyles that do not recognize the ``mcitetail'' field will silently ignore it.

Thus, for the example used in the introduction, the database entries might look like:
\begin{MYlistblk}
\MYlongcmd
@article{Glashow,|\
  author        = "Sheldon Lee Glashow",|\
  title         = "Partial-symmetries of Weak Interactions",|\
  journal       = "Nucl. Phys.",|\
  volume        = "22",|\
  number        = "4",|\
  month         = feb,|\
  year          = "1961",|\
  pages         = "579-588"|\
}|\
| |\
@incollection{Salam,|\
  author        = "Abdus Salam",|\
  editor        = "Nils Svartholm",|\
  title         = "Weak and Electromagnetic Interactions",|\
  booktitle     = "Elementary Particle Theory",|\
  publisher     = "Almquist and Wiksell",|\
  address       = "Stockholm",|\
  year          = "1968",|\
  pages         = "367-377",|\
  mcitetail     = "yes"|\
}|\
| |\
@article{Weinberg,|\
  author        = "Steven Weinberg",|\
  title         = "A Model of Leptons",|\
  journal       = "Phys. Rev. Lett.",|\
  volume        = "19",|\
  number        = "21",|\
  month         = nov,|\
  year          = "1967",|\
  pages         = "1264-1266",|\
  mcitetail     = "yes"|\
}|end
\end{MYlistblk}
The basic idea is for \BibTeX\ to use the sorting key of the head,
but with an appended tail count, as the sorting key for all the tails
of that head. If the bibstyle properly generates unique sorting keys, then the
entries in each group will be kept together and in order despite
the sorting process. For example, apsrmpM.bst uses the following
sort keys for the above three entries:
\begin{MYlistblk}
\MYlongcmd
glashow  s l    1961    partial symmetries of weak interactions|\
glashow  s l    1961    partial symmetries of weak interactions|_|_0000000001|\
glashow  s l    1961    partial symmetries of weak interactions|_|_0000000002|end
\end{MYlistblk}
The sorting modification to the bibstyle consists of three parts which must be
done in addition to the mciteplus compatibility modifications previously
described.

First of all, add an ``mcitetail'' field to the list of entry fields (usually
located near the beginning of the bibstyle):
\begin{MYlistblk}
\MYlongcmd
ENTRY|\
  { address|\
    archive|\
    author|\
    booktitle|\
    .|\
    .|\
    key|\
    mcitetail|\
    month|\
    .|\
    .|end
\end{MYlistblk}

Then, add the following code (which you can cut and paste from \MYsmalltt{mciteplus\_code.txt})
somewhere near the start of the bibstyle code (after the first ``STRINGS''
definition(s) is fine):
\begin{MYlistblk}
\MYlongcmd
|% mciteplus mcitetail field and sort key adjust support|\
INTEGERS {mcitetailcnt is.mcitetail}|\
STRINGS  {mciteheadsortkey}|\
|% convert the strings "yes" or "no" to |#1 or |#0 respectively|\
FUNCTION {mciteplus.yes.no.to.int}|\
{ "l" change.case$ duplicate$|\
    "yes" =|\
    { pop$ |#1 }|\
    { duplicate$ "no" =|\
        { pop$ |#0 }|\
        { "unknown boolean " quote$ * swap$ * quote$ *|\
          " in " * cite$ * warning$|\
          |#0|\
        }|\
      if$|\
    }|\
  if$|\
}|\
FUNCTION {mciteplustail.adj.sort.key}|\
{ mcitetail|\
  empty$|\
    { |#0 'is.mcitetail :=  }|\
    { mcitetail|\
      mciteplus.yes.no.to.int|\
      'is.mcitetail :=|\
    }|\
  if$|\
  is.mcitetail|\
    { |#1 mcitetailcnt +|\
      'mcitetailcnt :=|\
      mciteheadsortkey|\
      "|_|_" *|\
      "000000000"|\
      mcitetailcnt|\
      int.to.str$|\
      *|\
      |#-1 |#10 substring$|\
      *|\
      'sort.key$ :=|\
    }|\
    { |#0 'mcitetailcnt :=|\
      sort.key$ 'mciteheadsortkey :=|\
    }|\
  if$|\
}|\
|% END mciteplus mcitetail field support|end
\end{MYlistblk}

Lastly, you have to locate the place(s) where the
bibstyle makes the (final) assignment to \MYshortcmd+sort.key$|end\ 
(typically of the form \MYshortcmd+'sort.key$| :=|end\ ) before
sorting is performed and add a call to \MYshortcmd+mciteplustail.adj.sort.key|end\ 
(which will modify the sort keys of the tails as needed to keep the groups
together) just after it. With apsrmpM.bst, this is at the end of the
\MYshortcmd+FUNCTION {presort}|end:
\begin{MYlistblk}
\MYlongcmd
.|\
.|\
  'sort.key$ :=|\
  mciteplustail.adj.sort.key|\
}|end
\end{MYlistblk}
as well as at the end of \MYshortcmd+FUNCTION {bib.sort.order}|end.




\section{Advanced Usage}
The vast majority of users will not need to know the details of the internal
operation of mciteplus. However, the following information may be of use to
those who are interested in it or who need to interface other packages to mciteplus.

Mciteplus consists of two main parts: (1) a part that intercepts and records the
status of entries the user has cited; and (2) a part that acts on this
information when formatting the entries in the bibliography.

Upon the start of the \MYshortcmd\begin{document}|end, mciteplus stores the original
\MYshortcmd\cite|end\ and \MYshortcmd\nocite|end\ as \MYshortcmd\mciteOrgcite|end\ 
and \MYshortcmd\mciteOrgnocite|end, respectively, and replaces the originals with
its own versions. The mciteplus versions of the citation commands: (1) process the citation
list, recording which entries are heads and which are tails; (2) writes out the
citation list to the auxiliary file, much like \MYshortcmd\nocite|end\ does, to ensure
that all the given entries will appear in the bibliography as well as being in the correct
order; and (3) forwards the list of heads in the citation list to the original cite
command. Thus, \LaTeX\ remains unaware that the tail entries were ever cited.

The second part of mciteplus operates inside the
\MYshortcmd+mcitethebibliography|end\ environment where mciteplus: (1) intercepts
and adjusts the bibliography sample label as needed; (2) forwards to
the original \MYshortcmd+thebibliography|end; and (3) places a wrapper
around \MYshortcmd\bibitem|end, storing the original \MYshortcmd\bibitem|end\ as
\MYshortcmd\mciteOrgbibitem|end. For head entries, the mciteplus
\MYshortcmd\bibitem|end\ behaves in fairly normal fashion and forwards
to the original \MYshortcmd\bibitem|end. However, for tail entries, the
mciteplus \MYshortcmd\bibitem|end, does not forward them to the original
\MYshortcmd\bibitem|end. Thus, the tail entries become part of the
previous entry. To \LaTeX, it is as if the \MYshortcmd\bibitem|end\ of the
tails never happened.


\subsection{Tracking ID and Aux. Files}
\label{sec:trackingandaux}
Things get more complicated when a document has more than one bibliography. 
To support multiple bibliographies, mciteplus uses a ``tracking ID'' which
is simply an identification string which is unique to each bibliography.
Thus, the same citation key can be used in different ways in different
bibliographies (e.g., serving as a head in one and a tail in another)
without conflict because each instance will be considered different
as their tracking IDs are different.\footnote{Of course, a ``status conflict''
is still possible for ``overall'' bibliographies.}

The standard tracking ID is simply the string ``main'' which is carried as:
\begin{MYlistblk}
\MYlongcmd
\def\mcitetrackID{main}|\
\def\mcitebibtrackID{main}|end
\end{MYlistblk}
the former of which is used by the mciteplus citation commands and the
latter of which is used by the \MYshortcmd+mcitethebibliography|end\ environment.
However, these definitions may be automatically altered depending on the specific
needs of other \LaTeX\ packages that have been loaded
(\MYsecref{sec:externalpackages}). Some packages that provide support for
multiple bibliographies require that the tracking ID be closely coupled
with the individual citation commands. In such cases, mciteplus may not
even refer to \MYshortcmd\mcitetrackID|end\ or \MYshortcmd\mcitebibtrackID|end.

For the normal part/chapter bibliographies of chapterbib.sty, the tracking ID form
``chapterbib\textit{inputfile}'' is used, where ``\textit{inputfile}'' is
the include file (without the .tex suffix) the given citation occurred in.
The tracking ID of the duplicate bibliographies of chapterbib (produced by
chapterbib's ``duplicate'' package option) is given by ``chapterbib\textit{inputfile}.bbl''.
For the rootbib bibliography (produced by chapterbib's ``rootbib'' package option),
the tracking ID is of the form ``chapterbib\textit{jobname}'', where
``\textit{jobname}'' is the name of the main .tex document file (without the
.tex suffix).

For multibib.sty, the tracking ID is given by ``multibib\textit{secname}'', where
``\textit{secname}'' is the multibib bibliography section name (as declared
via \MYshortcmd\newcites|end).

For multibbl.sty, the tracking ID is given by ``multibbl\textit{secname}'', where
``\textit{secname}'' is the multibbl bibliography section name.

For bibunits.sty, the tracking ID of each bibunit is given by
``bibunits\textit{unitname}'', where ``\textit{unitname}'' is the name of the
auxiliary file of the desired bibunit without the .aux suffix
(e.g., ``bu1'', ``bu2'', etc.). Outside of the bibunits, where the global
bibliography is in effect, the tracking ID of ``main'' is used.

Likewise, mciteplus must know which auxiliary file(s) to write its citation lists
and bibliography maximum widths and counts to. The standard auxiliary file handles, which
are simply integers that represent files \TeX\ has opened, are defined as:
\begin{MYlistblk}
\MYlongcmd
\def\mciteauxout{\@auxout}|\
\def\mcitebibauxout{\@mainaux}|end
\end{MYlistblk}
the former of which is used by the mciteplus citation commands and the
latter of which is used by the \MYshortcmd+mcitethebibliography|end\ environment.
As with the tracking ID, these may be altered to support other packages
that have been loaded. Under some packages, such as chapterbib, \LaTeX\ may
reuse the same file handle number during the course of processing the document.
Thus, a given file handle number may not refer to the same file in
different parts of the document.

Note that mciteplus writes all of its maximum width and count
information to the main auxiliary file even though multiple
auxiliary files may be in use because it is not certain
that the other auxiliary files will be read in during \LaTeX\ runs.
As the tracking ID is written along with the maximum width and count information,
there will be no confusion as to which bibliography the information pertains to.


\subsection{Custom Command Wrappers}
\label{sec:customwrappers}
Mciteplus provides two rather complex commands that allow users to define their
own mciteplus citation command wrappers:
\begin{MYlistblk}
\MYlongcmd
\mciteCiteA*{|itshape+aux out|upshape}{|itshape+track ID|upshape}{|itshape+prehandler|upshape}{|itshape+posthandler|upshape}{|itshape+fwd|upshape}*[|itshape+opt1|upshape][|itshape+opt2|upshape]{|itshape+cite list|upshape}|\
\mciteCiteB*{|itshape+aux out|upshape}{|itshape+track ID|upshape}{|itshape+prehandler|upshape}{|itshape+posthandler|upshape}{|itshape+fwd|upshape}*[|itshape+opt1|upshape][|itshape+opt2|upshape]{|itshape+sec ID|upshape}{|itshape+cite list|upshape}|end
\end{MYlistblk}
Both of these are ``robust'' and can safely be called within footnotes,
etc. The difference between the two is that the ``B'' form supports citation commands
that use two arguments, as is the case with the \MYshortcmd\cite|end\ of
multibbl.sty. \MYshortcmd\mciteCiteA/B|end\ acquires all of the arguments
listed above (the ``*'' and ``[]'' are optional arguments), but those that appear
after the \textit{fwd} argument are forwarded (possibly with modification)
to the command named in the \textit{fwd} argument. The arguments before
\textit{fwd} are only seen and used by \MYshortcmd\mciteCiteA/B|end.

The meaning of each of the options from left to right is as follows:
\begin{description}
\item[*] The star forms (e.g., \MYshortcmd\mciteCiteA/B*|end) disable automatic
internal processing (via \MYshortcmd\mciteDoList|end\ as discussed below).
The presence of the star form is indicated by the \TeX\ conditional
\MYshortcmd\ifmciteMacroStarForm|end.
\item[\textit{aux out}] The auxiliary file handle number to be used when
writing out the citation list. Use of the special string ``noauxwrite'' will
disable auxiliary writes.\footnote{As will the traditional \TeX\ conditional,
which is defined and used by \LaTeX, \texttt{\mbox{\ZAXbsl if@filesw}}, if it is false.}
This argument is stored in the macro \MYshortcmd\mciteCiteAuxArg|end.
\item[\textit{track ID}] The tracking ID to be used. This argument is
stored in the macro \MYshortcmd\mciteCiteTrackArg|end.
\item[\textit{prehandler}] The user's prehandler code which will be
executed before internal \MYshortcmd\mciteDoList|end\ processing. This
argument is stored in the macro \MYshortcmd\mciteCitePrehandlerArg|end. 
The purpose of the prehandler code is to provide the user a way to alter 
the arguments as needed before further processing.
\item[\textit{posthandler}] The user's posthandler code which will be
executed after internal \MYshortcmd\mciteDoList|end\ processing. This
argument is stored in the macro \MYshortcmd\mciteCitePosthandlerArg|end.
\item[\textit{fwd}] The citation command the later arguments will be forwarded
to. This argument is stored in the macro \MYshortcmd\mciteCiteFwdArg|end.
\item[*] The star for the star form of the forward command.
The presence of the star form is indicated by the \TeX\ conditional
\MYshortcmd\ifmciteCiteStarFwdArg|end.
\item[\textit{opt1}] The first optional argument of the forwarded citation
command. The presence of this optional argument is
indicated by the \TeX\ conditional \MYshortcmd\ifmciteMacroOptArgI|end.
This argument, if present, is stored in the macro \MYshortcmd\mciteCiteOptArgI|end.
\item[\textit{opt2}] The second optional argument of the forwarded citation
command. The presence of this optional argument is
indicated by the \TeX\ conditional \MYshortcmd\ifmciteMacroOptArgII|end.
This argument, if present, is stored in the macro \MYshortcmd\mciteCiteOptArgII|end.
\item[\textit{sec ID}] Only used for \MYshortcmd\mciteCiteB|end, this
is the first argument of two used by the forwarded citation command.
Under multibbl.sty, this contains the bibliography section name.
This argument is stored in the macro \MYshortcmd\mciteCiteSecIDArg|end.
\item[\textit{cite list}] This is the citation list and is
stored in the macro \MYshortcmd\mciteCiteListArg|end. It is important
to know that \MYshortcmd\mciteCiteListArg|end\ is \emph{not} forwarded
as-is, but rather only after the tails are removed. The forwarded
citation list is the macro \MYshortcmd\mciteFwdCiteListArg|end.
\end{description}
\subsubsection{The MciteDoList Engine}
\label{sec:mcitedolistengine}
After acquiring its arguments, \MYshortcmd\mciteCiteA/B|end\ executes
the prehandler code. This is to allow the prehandler code to alter the
arguments as needed, specifically if the contents of one argument need
to be based on another. For example, under multibbl.sty, the tracking
ID has to be based in part on \MYshortcmd\mciteCiteSecIDArg|end.

After the prehandler, the \MYshortcmd\mciteDoList|end\
command, which forms the core engine of mciteplus, is executed:
\begin{MYlistblk}
\MYlongcmd
\mciteDoList{|itshape+aux out|upshape}{|itshape+track ID|upshape}{|itshape+cite list|upshape}|end
\end{MYlistblk}
as:
\begin{MYlistblk}
\MYlongcmd
\mciteDoList{\mciteCiteAuxArg}{\mciteCiteTrackArg}{\mciteCiteListArg}|end
\end{MYlistblk}
\MYshortcmd\mciteDoList|end\ records the status of each of the entries, writes the
citations to the auxiliary file, and then creates \MYshortcmd\mciteheadlist|end\
which contains only the head entries. A copy of \MYshortcmd\mciteheadlist|end\ 
called \MYshortcmd\mciteFwdCiteListArg|end\ is made. This will be 
the actual citation list that is forwarded to the \textit{fwd} command.

Next, \MYshortcmd\mciteExtraDoLists|end\ is
called. This command is normally, just \MYshortcmd\relax|end, but it can be
used to create additional sets of entry status, such as when duplicate/global
bibliographies are being used.

Then, the user's posthandler code is executed.

Finally, the forward command is executed with the (rightmost) 
\MYshortcmd\mciteCiteA/B|end\ arguments forwarded to it, but
using \MYshortcmd\mciteFwdCiteListArg|end\ for the citation list.

As a practical example, a standard \MYshortcmd\cite|end\ wrapper can be
created via:
\begin{MYlistblk}
\MYlongcmd
\newcommand{\MYcite}{\mciteCiteA{\mciteauxout}{\mcitetrackID}{\relax}{\relax}{\mciteOrgcite}}|end
\end{MYlistblk}
If \MYshortcmd\MYcite|end\ is invoked as:
\begin{MYlistblk}
\MYlongcmd
\MYcite{Glashow,*Salam,*Weinberg}|end
\end{MYlistblk}
mciteplus will do a:
\begin{MYlistblk}
\MYlongcmd
\mciteCiteA{\mciteCiteAuxArg}{\mciteCiteTrackArg}{\relax}{\relax}{\mciteOrgcite}{\mciteCiteListArg}|end
\end{MYlistblk}
which is the same as:
\begin{MYlistblk}
\MYlongcmd
\mciteCiteA{\mciteauxout}{\mcitetrackID}{\relax}{\relax}{\mciteOrgcite}{Glashow,*Salam,*Weinberg}|end
\end{MYlistblk}
which in turn would call:
\begin{MYlistblk}
\MYlongcmd
\mciteDoList{\mciteauxout}{\mcitetrackID}{Glashow,*Salam,*Weinberg}|end
\end{MYlistblk}
which will create a \MYshortcmd\mciteheadlist|end\ and
\MYshortcmd\mciteFwdCiteListArg|end\ containing ``Glashow'', which
will then be forwarded to the original \LaTeX\ \MYshortcmd\cite|end:
\begin{MYlistblk}
\MYlongcmd
\mciteOrgcite{\mciteFwdCiteListArg}|end
\end{MYlistblk}
which is effectively the same as:
\begin{MYlistblk}
\MYlongcmd
\mciteOrgcite{Glashow}|end
\end{MYlistblk}
Mciteplus carefully builds up the forwarded arguments using token lists
so that the forward cite command isn't even aware that it has been
wrapped within an earlier command. Thus, even cite.sty's automatic
punctuation and spacing adjustments will work as normal.

Note that command structures that result in ``double calls'' to the miteplus
engine (i.e., a cite command goes through mciteplus and then forwards to
another cite command that also goes through mciteplus) although undesirable,
will still work as intended without error.

\subsection{Mcitethebibliography Hooks}
\label{sec:mcitethebibliographyhooks}
Mciteplus supplies the macros \MYshortcmd\mciteBIBdecl|end\ and
\MYshortcmd\mciteBIBenddecl|end\ 
which are executed at the beginning and ending of the
\MYshortcmd+mcitethebibliography|end\ environment, respectively, before the the normal
\MYshortcmd+thebibliography|end\ environment is begun or ended. By default they are
both defined as macros containing only \MYshortcmd\relax|end, but a
user may redefine them as needed to alter the
\MYshortcmd+mcitethebibliography|end\ environment setup.

There is also a \MYshortcmd\mcitefwdBIBdecl|end\ which is executed
just after the original \MYshortcmd+thebibliography|end\ has been
started (i.e., forwarded) by \MYshortcmd+mcitethebibliography|end.
This can be used to adjust the setup of \MYshortcmd+thebibliography|end\ 
before the first bibliography entry. One potential application is
to sync the \MYshortcmd+thebibliography|end\ counter to
\MYshortcmd+mcitebibitemcount|end. This trick can be used to obtain
continuous numbering of the references under multibbl.sty even though 
that package does not normally provide this feature:
\begin{MYlistblk}
\MYlongcmd
\documentclass{article}|\
\usepackage{multibbl}|\
\usepackage{mciteplus}|\
\mciteResetBibitemCountfalse|\
\renewcommand{\mcitefwdBIBdecl}{\setcounter{enumiv}{\value{mcitebibitemcount}}}|\
\begin{document}|\
.|\
.|end
\end{MYlistblk}




\section{Use With External Packages}
\label{sec:externalpackages}
Unless the ``nohooks'' option is invoked, mciteplus will automatically
reconfigure itself and hook into the compatible packages it detects.
Mciteplus should be loaded after other packages so that it can detect
the other packages and to ensure that the definitions of the system
cite commands have been finalized before mciteplus installs its
wrappers at the beginning of the document.

It is not possible to test or list all of the possible interactions
among all the various packages. Remember, just because the packages
below are compatible with mciteplus, does not mean they all are
compatible with each other, even in the absence of mciteplus. Here
are some notes regarding the compatibility
and use of mciteplus with some of the more common \LaTeX\ 
bibliography related packages.



\subsection{Bibunits}
\label{sec:bibunits}
Mciteplus is fully compatible with Thorsten Hansen's bibunits
package \cite{ctan:bibunits}. When using the global bibliography via
the ``globalcitecopy'' and/or the star form of \MYshortcmd\cite|end,
beware of the possibility of an mciteplus status (i.e., head or tail)
conflict if the same entry is cited differently in different parts of
the document.


\subsection{Chapterbib}
\label{sec:chapterbib}
Mciteplus is fully compatible with Donald Arseneau's chapterbib
package \cite{ctan:chapterbib} including the use of the chapterbib 
``duplicate'' and ``rootbib'' package options. \textbf{When using the
``rootbib'' option, be sure and enable mciteplus' 
``chapterbibrootbib'' option.} Mciteplus is not able to auto-detect
the rootbib mode of chapterbib directly because when invoking that mode, 
the user must run \LaTeX\ a second time without the rootbib option.
For all of these \LaTeX\ runs, keep this mciteplus option enabled (i.e.,
as long as there is, or is going to be, a jobname.bbl file under chapterbib).
When using the rootbib option, beware of the possibility of an mciteplus
status (i.e., head or tail) conflict if the same entry is cited differently
in different parts of the document.


\subsection{Cite}
\label{sec:cite}
Mciteplus is fully compatible with Donald Arseneau's cite
package \cite{ctan:citesty}, including cite's \MYshortcmd\citen|end,
\MYshortcmd\citenum|end\ and \MYshortcmd\citeonline|end\
variants.


\subsection{Citeref}
\label{sec:citeref}
Mciteplus will work with Bj\"{o}rn Briel and Uni Oldenburg's citeref
package \cite{ctan:citeref}.


\subsection{Drftcite}
\label{sec:drftcite}
Mciteplus is fully compatible with Donald Arseneau's 
drftcite package \cite{ctan:drftcite}, including drftcite's
\MYshortcmd\citen|end\ variant. To get block aligned entry text
in the bibliography, set the bibitem maximum width form
to use the citation keys drftcite uses for the bibliography labels:
\begin{MYlistblk}
\MYlongcmd
\mciteSetMaxWidthForm{bibitem}{\mciteBibitemArgI}|end
\end{MYlistblk}
which, in turn, will revise the bibliography sample label
accordingly. This is not done automatically by mciteplus because
drftcite does not update the bibliography sample label (even
in the absence of mciteplus).


\subsection{Hyperref/Backref}
\label{sec:hyperrefbackref}
Mciteplus should work fine with Sebastian Rahtz and Heiko Oberdiek's
hyperref package \cite{ctan:hyperref} including it's backref option.

\subsection{Multibbl}
\label{sec:multibbl}
Mciteplus is fully compatible with Apostolos Syropoulos's 
multibbl package \cite{ctan:multibbl}.


\subsection{Multibib}
\label{sec:multibib}
Mciteplus is fully compatible with Thorsten Hansen's multibib
package \cite{ctan:multibib}. Under mciteplus with natbib,
multibib is patched to support all of natbib's \MYshortcmd\cite|end\ 
variants even though the original multibib only supported
\MYshortcmd\citep|end, \MYshortcmd\citet|end, \MYshortcmd\citealp|end\ and 
\MYshortcmd\citealt|end. Multibib's internal command name list hook
\MYshortcmd\@mb@citenamelist|end\ is ignored.


\subsection{Natbib}
\label{sec:natbib}
Mciteplus is fully compatible with Patrick W. Daly's natbib package
\cite{ctan:natbib} as long as the bibstyle supports both.
All of natbib's \MYshortcmd\cite|end\ variants are supported
including: 
\MYshortcmd\citep|end, \MYshortcmd\Citep|end, \MYshortcmd\citet|end, 
\MYshortcmd\Citet|end, \MYshortcmd\citealp|end, \MYshortcmd\Citealp|end, 
\MYshortcmd\citealt|end, \MYshortcmd\Citealt|end, \MYshortcmd\citeauthor|end, 
\MYshortcmd\Citeauthor|end, \MYshortcmd\citeyear|end, \MYshortcmd\citeyearpar|end,
\MYshortcmd\citepalias|end\ and \MYshortcmd\citetalias|end.


\subsection{Notes2bib}
\label{sec:notes2bib}
Mciteplus is fully compatible with Joseph Wright's notes2bib package \cite{ctan:notes2bib}.
However, version 1.3 (January 2008) or later must be used for the notes2bib
 ``tail'' or ``head'' options to work with mciteplus.


\subsection{REV\TeX}
\label{sec:revtex}
Mciteplus is fully compatible with Arthur Ogawa and David Carlisle's REV\TeX\ (Version 4)
class \cite{ctan:revtex}, including support for end notes in the bibliography.
Note that to support this feature, REV\TeX\ uses its own internal bibliography sample
label and ignores the ones provided by \BibTeX\ and mciteplus. One annoyance is
that REV\TeX\ writes the footnote ``citations'' to the auxiliary file which causes
\BibTeX\ to complain that it can't find these ``entries'' in its database (e.g.,
``\texttt{Warning--I didn't find a database entry for "endnote7"}''. To stop this
problem, add the following patch code (which you can cut and paste from \MYsmalltt{mciteplus\_code.txt})
right after REV\TeX\ is loaded:
\begin{MYlistblk}
\MYlongcmd
|% Patch REVTeX to prevent BibTeX from seeing endnotes as citations|\
|% Insert just after REVTeX is loaded|\
\makeatletter|\
\let\@ORGREVTEXendnotemark\@endnotemark|\
\let\@ORGREVTEX@makefnmark@cite\@makefnmark@cite|\
\def\@endnotemark{\bgroup\@fileswfalse\@ORGREVTEXendnotemark\egroup}|\
\def\@makefnmark@cite{\bgroup\@fileswfalse\@ORGREVTEX@makefnmark@cite\egroup}|\
\makeatother|end
\end{MYlistblk}


\subsection{Partially Supported Packages}
\label{sec:partiallysupportedpackages}

\noindent\textbf{Footbib:} Eric Domenjoud's footbib package \cite{ctan:footbib}
``peacefully coexists'' with mciteplus in the sense that they do not interact.
At present, an mciteplus compatible bibstyle cannot be used for the footbibliography.


\subsection{Incompatible Packages}
\label{sec:incompatiblepackages}
Unfortunately, the following packages are currently known not to work with mciteplus.
Many of them clash with mciteplus on a fundamental level. However, others may be
supported in the future.

\noindent\textbf{Amsrefs:} David Jones' amsrefs package \cite{ctan:amsrefs}.

\noindent\textbf{Apacite:} Erik Meijer's apacite package \cite{ctan:apacite}.

\noindent\textbf{Biblatex:} Philipp Lehman's biblatex package \cite{ctan:biblatex}.
Future versions of biblatex may well offer features like those of mciteplus.

\noindent\textbf{Bibtopic:} Stefan Ulrich and Pierre Basso's bibtopic package \cite{ctan:bibtopic}.

\noindent\textbf{Inlinebib:} Ren{\'e} Seindal's inlinebib package \cite{ctan:inlinebib}.

\noindent\textbf{Jurabib:} Jens Berger's jurabib package \cite{ctan:jurabib}.

\noindent\textbf{Opcit:} Federico Garcia's opcit package \cite{ctan:opcit}.

\noindent\textbf{Splitbib:} Nicolas Markey's splitbib package \cite{ctan:splitbib}.


\section*{Acknowledgments}
\addcontentsline{toc}{section}{Acknowledgments}
First and foremost, the author would like to thank Joseph Wright.
Joseph beta tested prototype versions as well as provided many
ideas that improved the final architecture of mciteplus.

Thorsten Ohl's mcite package identified the need for a \LaTeX\
package that provides this type of bibliography handling and set
the standard for so doing. Although coded differently, mciteplus
uses the same basic approach to the problem (i.e., wrap the
citation command, drop the tails, forward the head list to the
original cite command, and selectively drop items in the
bibliography) as originally implemented in mcite.sty.


\newcommand{\BIBdecl}{\addcontentsline{toc}{section}{\refname}}



%\bibliographystyle{IEEEtran}
%\bibliography{mciteplus_doc}

% Generated by IEEEtran.bst, version: 1.13 (2008/09/30)
\begin{thebibliography}{10}
\providecommand{\url}[1]{#1}
\csname url@samestyle\endcsname
\providecommand{\newblock}{\relax}
\providecommand{\bibinfo}[2]{#2}
\providecommand{\BIBentrySTDinterwordspacing}{\spaceskip=0pt\relax}
\providecommand{\BIBentryALTinterwordstretchfactor}{4}
\providecommand{\BIBentryALTinterwordspacing}{\spaceskip=\fontdimen2\font plus
\BIBentryALTinterwordstretchfactor\fontdimen3\font minus
  \fontdimen4\font\relax}
\providecommand{\BIBforeignlanguage}[2]{{%
\expandafter\ifx\csname l@#1\endcsname\relax
\typeout{** WARNING: IEEEtran.bst: No hyphenation pattern has been}%
\typeout{** loaded for the language `#1'. Using the pattern for}%
\typeout{** the default language instead.}%
\else
\language=\csname l@#1\endcsname
\fi
#2}}
\providecommand{\BIBdecl}{\relax}
\BIBdecl
\renewcommand{\BIBentryALTinterwordstretchfactor}{4}

\bibitem{ctan:mciteplus}
\BIBentryALTinterwordspacing
M.~Shell. (2008, Sep.) The mciteplus.sty package. [Online]. Available:
  \url{http://www.ctan.org/tex-archive/macros/latex/contrib/mciteplus/}
\BIBentrySTDinterwordspacing

\bibitem{ctan:mcite}
\BIBentryALTinterwordspacing
T.~Ohl. (1996, Jan.) The mcite.sty package. [Online]. Available:
  \url{http://www.ctan.org/tex-archive/macros/latex/contrib/mcite/}
\BIBentrySTDinterwordspacing

\bibitem{ctan:revtex}
\BIBentryALTinterwordspacing
A.~Ogawa and D.~Carlisle. (2001, Aug.) The {REV\TeX} package. [Online].
  Available: \url{http://www.ctan.org/tex-archive/macros/latex/contrib/revtex/}
\BIBentrySTDinterwordspacing

\bibitem{ctan:ieeetranbst}
\BIBentryALTinterwordspacing
M.~Shell. (2008, Sep.) The {IEEE}tran {{\BibTeX}} style. [Online]. Available:
  \url{http://www.ctan.org/tex-archive/macros/latex/contrib/IEEEtran/bibtex/}
\BIBentrySTDinterwordspacing

\bibitem{ctan:hyperref}
\BIBentryALTinterwordspacing
S.~Rahtz and H.~Oberdiek. (2008, Sep.) The hyperref.sty package. [Online].
  Available:
  \url{http://www.ctan.org/tex-archive/macros/latex/contrib/hyperref/}
\BIBentrySTDinterwordspacing

\bibitem{ctan:chapterbib}
\BIBentryALTinterwordspacing
D.~Arseneau. (2004, Mar.) The chapterbib.sty package. [Online]. Available:
  \url{http://www.ctan.org/tex-archive/macros/latex/contrib/cite/}
\BIBentrySTDinterwordspacing

\bibitem{ctan:eqparbox}
\BIBentryALTinterwordspacing
S.~Pakin. (2004, Aug.) The eqparbox.sty package. [Online]. Available:
  \url{http://www.ctan.org/tex-archive/macros/latex/contrib/eqparbox/}
\BIBentrySTDinterwordspacing

\bibitem{ctan:alphabst}
\BIBentryALTinterwordspacing
O.~Patashnik. (1988, Feb.) The alpha.bst bibstyle. [Online]. Available:
  \url{http://www.ctan.org/tex-archive/biblio/bibtex/distribs/styles/}
\BIBentrySTDinterwordspacing

\bibitem{ctan:amsrefs}
\BIBentryALTinterwordspacing
M.~Downes and D.~M. Jones. (2007, Oct.) The amsrefs package. [Online].
  Available:
  \url{http://www.ctan.org/tex-archive/macros/latex/contrib/amsrefs/}
\BIBentrySTDinterwordspacing

\bibitem{ctan:biblatex}
\BIBentryALTinterwordspacing
P.~Lehman. (2008, Apr.) The biblatex package. [Online]. Available:
  \url{http://www.ctan.org/tex-archive/macros/latex/exptl/biblatex/}
\BIBentrySTDinterwordspacing

\bibitem{ctan:bibunits}
\BIBentryALTinterwordspacing
T.~Hansen. (2004, May) The bibunits.sty package. [Online]. Available:
  \url{http://www.ctan.org/tex-archive/macros/latex/contrib/bibunits/}
\BIBentrySTDinterwordspacing

\bibitem{ctan:citesty}
\BIBentryALTinterwordspacing
D.~Arseneau. (2006, Sep.) The cite.sty package. [Online]. Available:
  \url{http://www.ctan.org/tex-archive/macros/latex/contrib/cite/}
\BIBentrySTDinterwordspacing

\bibitem{ctan:citeref}
\BIBentryALTinterwordspacing
B.~Briel and U.~Oldenburg. (1999, May) The citeref.sty package. [Online].
  Available:
  \url{http://www.ctan.org/tex-archive/macros/latex/contrib/citeref/citeref.st%
y}
\BIBentrySTDinterwordspacing

\bibitem{ctan:drftcite}
\BIBentryALTinterwordspacing
D.~Arseneau. (2007, Dec.) The drftcite.sty package. [Online]. Available:
  \url{http://www.ctan.org/tex-archive/macros/latex/contrib/cite/}
\BIBentrySTDinterwordspacing

\bibitem{ctan:multibbl}
\BIBentryALTinterwordspacing
A.~Syropoulos. (2004, Jul.) The multibbl.sty package. [Online]. Available:
  \url{http://www.ctan.org/tex-archive/macros/latex/contrib/multibbl/}
\BIBentrySTDinterwordspacing

\bibitem{ctan:multibib}
\BIBentryALTinterwordspacing
T.~Hansen. (2004, Jan.) The multibib.sty package. [Online]. Available:
  \url{http://www.ctan.org/tex-archive/macros/latex/contrib/multibib/}
\BIBentrySTDinterwordspacing

\bibitem{ctan:natbib}
\BIBentryALTinterwordspacing
P.~W. Daly. (2007, Oct.) The natbib.sty package. [Online]. Available:
  \url{http://www.ctan.org/tex-archive/macros/latex/contrib/natbib/}
\BIBentrySTDinterwordspacing

\bibitem{ctan:notes2bib}
\BIBentryALTinterwordspacing
J.~Wright. (2008, Sep.) The notes2bib.sty package. [Online]. Available:
  \url{http://www.ctan.org/tex-archive/macros/latex/contrib/notes2bib/}
\BIBentrySTDinterwordspacing

\bibitem{ctan:footbib}
\BIBentryALTinterwordspacing
E.~Domenjoud. (2005, May) The footbib.sty package. [Online]. Available:
  \url{http://www.ctan.org/tex-archive/macros/latex/contrib/footbib/}
\BIBentrySTDinterwordspacing

\bibitem{ctan:apacite}
\BIBentryALTinterwordspacing
E.~Meijer. (2007, Sep.) The apacite package. [Online]. Available:
  \url{http://www.ctan.org/tex-archive/biblio/bibtex/contrib/apacite/}
\BIBentrySTDinterwordspacing

\bibitem{ctan:bibtopic}
\BIBentryALTinterwordspacing
S.~Ulrich and P.~Basso. (2006, Sep.) The bibtopic.sty package. [Online].
  Available:
  \url{http://www.ctan.org/tex-archive/macros/latex/contrib/bibtopic/}
\BIBentrySTDinterwordspacing

\bibitem{ctan:inlinebib}
\BIBentryALTinterwordspacing
R.~Seindal. (1999, Jul.) The inlinebib.sty package. [Online]. Available:
  \url{http://www.ctan.org/tex-archive/biblio/bibtex/contrib/inlinebib/}
\BIBentrySTDinterwordspacing

\bibitem{ctan:jurabib}
\BIBentryALTinterwordspacing
J.~Berger. (2004, Jan.) The jurabib.sty package. [Online]. Available:
  \url{http://www.ctan.org/tex-archive/macros/latex/contrib/jurabib/}
\BIBentrySTDinterwordspacing

\bibitem{ctan:opcit}
\BIBentryALTinterwordspacing
F.~Garcia. (2007, Jun.) The opcit.sty package. [Online]. Available:
  \url{http://www.ctan.org/tex-archive/macros/latex/contrib/opcit/}
\BIBentrySTDinterwordspacing

\bibitem{ctan:splitbib}
\BIBentryALTinterwordspacing
N.~Markey. (2005, Dec.) The splitbib.sty package. [Online]. Available:
  \url{http://www.ctan.org/tex-archive/macros/latex/contrib/splitbib/}
\BIBentrySTDinterwordspacing

\end{thebibliography}







\end{document}
